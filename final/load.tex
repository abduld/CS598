
\section{Trepn Tool}

Since Android does not offer a way to capture processor usage information
  programatically, we use the Trepn tool by Qualqomm to capture the
  data for us and save it into a {\tt csv} file. 
To reduce overhead, Trepn measure the processor usage information every $100ms$, both
  the frequency and the load are measured sequentially, we therefore need to 
  correct that when parsing the {\tt csv} file.
First, we parse each processor reading along with the time stamp for reading
  the file.
Next, we interpolate the measured data (we use linear interpolant) and
  evaluate the interpolant at the application state times (these are the
  times Trepn recieved a signal from our application and correspond to
  timed blocks of code).
We then multiply the load by the frequency, and rescale all the CPU and GPU data (we perform the rescaling on the CPU and GPU seperatly).
Trepn can have measurment errors, resulting in infinite numbers.
To make sure that these do not skew the plots, we clip the range of possible processor reading to be between 1 and the the $0.99$th quantile of the data.
While efforts have been taken to reduce the profiler's overhead, it still 
  ocupies around a $10\%$ overhead.