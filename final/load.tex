\subsection{Processor Utilization}


\begin{figure*}[ht]
  \centering

  \begin{subfigure}[b]{\textwidth}
          \centering
          \includegraphics[width=0.4\textwidth]{data/load_legend.pdf}
  \end{subfigure}

  \begin{subfigure}[b]{0.9\textwidth}
       \centering
       \includegraphics[width=\textwidth]{data/load_vectoradd_nexus5.pdf}
       \caption{VectorAdd on Nexus 5}\label{fig:Vecadd5}
   \end{subfigure}
  \begin{subfigure}[b]{0.9\textwidth}
       \centering
       \includegraphics[width=\textwidth]{data/load_vectoradd_nexus7.pdf}
       \caption{VectorAdd on Nexus 7}\label{fig:Vecadd7}
   \end{subfigure}

  \begin{subfigure}[b]{0.9\textwidth}
       \centering
       \includegraphics[width=\textwidth]{data/load_sgemm_nexus5.pdf}
       \caption{Sgemm on Nexus 5}\label{fig:Sgemm5}
   \end{subfigure}
  \begin{subfigure}[b]{0.9\textwidth}
       \centering
       \includegraphics[width=\textwidth]{data/load_sgemm_nexus7.pdf}
       \caption{Sgemm on Nexus 7}\label{fig:Sgemm7}
   \end{subfigure}


  \caption{Processor utilization of VectorAdd and SGEMM for both Nexus 5 and Nexus 7. The Y axis is normalized across implementations with $1$ representing the peek utlization ($frequency*load$) for CPUs or GPUs. The X axis encodes timed blocked and is not scaled across implementations.}
\end{figure*}
\FloatBarrier

Processor utlization is measured by multiplying the $frequency$ and the $load$ information collected via Trepn.
The information gives us insight as to what parts of the code are active in each
	section of the code, how that impacts performance and battery.
In this section we discuss the results collected.

\subsubsection{VectorAdd}

VectorAdd, unlike other benchmarks, has a very high memory to compute ratio.
We therefore see very low 

\subsubsection{SGEMM}

\subsubsection{TPACF}

\subsubsection{MRIQ}

\subsubsection{Stencil}

\subsubsection{Histogram}

\begin{figure*}[ht]
  \centering

  \begin{subfigure}[b]{\textwidth}
          \centering
          \includegraphics[width=0.4\textwidth]{data/load_legend.pdf}
  \end{subfigure}

  \begin{subfigure}[b]{0.9\textwidth}
      \centering
      \includegraphics[width=\textwidth]{data/load_mriq_nexus5.pdf}
      \caption{MRIQ on Nexus 5}
      \label{fig:MRIQ5}
  \end{subfigure}
  \begin{subfigure}[b]{0.9\textwidth}
      \centering
      \includegraphics[width=\textwidth]{data/load_mriq_nexus7.pdf}
      \caption{MRIQ on Nexus 7}
      \label{fig:MRIQ7}
  \end{subfigure}

  \begin{subfigure}[b]{0.9\textwidth}
      \centering
      \includegraphics[width=\textwidth]{data/load_tpacf_nexus5.pdf}
      \caption{TPACF on Nexus 5}
      \label{fig:TPACF5}
  \end{subfigure}
  \begin{subfigure}[b]{0.9\textwidth}
      \centering
      \includegraphics[width=\textwidth]{data/load_tpacf_nexus7.pdf}
      \caption{TPACF on Nexus 7}
      \label{fig:TPACF7}
  \end{subfigure}

  \caption{Processor utilization of MRIQ and TPACF for both Nexus 5 and Nexus 7. The Y axis is normalized across implementations with $1$ representing the peek utlization ($frequency*load$) for CPUs or GPUs. The X axis encodes timed blocked and is not scaled across implementations.}
\end{figure*}
\FloatBarrier



\begin{figure*}[ht]
  \centering

  \begin{subfigure}[b]{\textwidth}
          \centering
          \includegraphics[width=0.4\textwidth]{data/load_legend.pdf}
  \end{subfigure}

  \begin{subfigure}[b]{0.9\textwidth}
      \centering
      \includegraphics[width=\textwidth]{data/load_histogram_nexus5.pdf}
      \caption{Histogram on Nexus 5}
      \label{fig:Histogram5}
  \end{subfigure}
  \begin{subfigure}[b]{0.9\textwidth}
      \centering
      \includegraphics[width=\textwidth]{data/load_stencil_nexus5.pdf}
      \caption{Stencil on Nexus 5}
      \label{fig:Stencil5}
  \end{subfigure}
  \end{subfigure}

  \caption{Processor utilization of Histogram and Stencil for Nexus 5. The Y axis is normalized across implementations with $1$ representing the peek utlization ($frequency*load$) for CPUs or GPUs. The X axis encodes timed blocked and is not scaled across implementations.}
\end{figure*}
\FloatBarrier