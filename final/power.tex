\subsection{Energy Utilization}


Since mobile devices employ DVFS,
  the energy utilization of the devices at a specific time is governed by
  the operating frequency of the processor.
One can model~\cite{zhang2010accurate, dong2011self} the battery usage
at time $t$ by using the following formula:

\begin{align*}
P(t) &= \beta_v(\beta(t)) + G_u(t) ~ G_v(G_f(t)) \\
     &+ \sum_{i=1}^{N} C_{u_n}(t) ~ C_{v_n}(C_{f_n}(t))
\end{align*}

where $N$ is the number of CPU cores, $\beta$ is the brightness of the LCD screen at time $t$, $\beta_v(br)$ is the power used for the specified brightness level, $G_f(t)$ and $C_{f_n}(t)$ are the operating frequencies at time $t$, $G_v(f)$ and $ C_{v_n}(f)$ are the power draws for the processors at the specified voltage, and $G_u(t)$ and $C_{u_n}(t)$ are the processor utilizations at time $t$.
Other terms, such as GPS, wireless, and other sensors, can be measured or modeled, but for this analysis we turned them off.

The issues with using a model is determining the power draw at the specified
  frequency (which is not specified by the processor's manufacturer) and the method of reading
  the CPU and GPU counters varies from device to device.
As a result, we again use Trepn to read the hardware counters giving us the battery usage (in $uW$) at each time interval.
Battery usage consists of all power-consuming components, e.g., screen, CPU, GPU, and memory.
If, for example, the computation takes longer then more power is consumed by other components, such as the screen.
Therefore, the energy consumed by a computation is not only dependent on of the load, but also the time it takes to perform a computation.
We place the device in airplane mode (to disable GPS, wireless, and other sensors) and disconnect the device from a power source to 
  collect the data.
Similar to the when measuring the processor utilization, we only consider the case where the kernel is executed many times.

\begin{figure}
  \centering

  \begin{subfigure}[b]{0.5\textwidth}
          \centering
          \includegraphics[width=0.6\textwidth]{data/legend.pdf}
  \end{subfigure}

  \begin{subfigure}[b]{0.23\textwidth}
      \centering
      \includegraphics[width=0.9\textwidth]{data/bbattery_vectoradd.pdf}
      \caption{VectorAdd}
  \end{subfigure}%
  \begin{subfigure}[b]{0.23\textwidth}
      \centering
      \includegraphics[width=0.9\textwidth]{data/bbattery_sgemm.pdf}
      \caption{Sgemm}
  \end{subfigure}


  \begin{subfigure}[b]{0.23\textwidth}
      \centering
      \includegraphics[width=0.9\textwidth]{data/bbattery_mriq.pdf}
      \caption{MRIQ}
  \end{subfigure}
  \begin{subfigure}[b]{0.23\textwidth}
      \centering
      \includegraphics[width=0.9\textwidth]{data/bbattery_tpacf.pdf}
      \caption{TPACF}
  \end{subfigure}%

  \begin{subfigure}[b]{0.23\textwidth}
      \centering
      \includegraphics[width=0.9\textwidth]{data/bbattery_histogram.pdf}
      \caption{Histogram}
  \end{subfigure}%
  \begin{subfigure}[b]{0.23\textwidth}
      \centering
      \includegraphics[width=0.9\textwidth]{data/bbattery_stencil.pdf}
      \caption{Stencil} 
  \end{subfigure}
  \caption{Energy usage for both the Nexus 5 and Nexus 7 normalized to Java's energy utilization (lower is better).}
  \label{fig:power}
\end{figure}

Figure~\ref{fig:power} shows the energy consumed by each implementation for both Nexus 5 and Nexus 7 
  (Trepn is not able to read hardware counters for the other devices).
While, as discussed in the previous section, RenderScript performs better, it does make full utilization of all CPU
  cores.
OpenCL, on the other hand, does not make use of all cores and only fully utilizes GPUs.
Since OpenCL has a similar time profile as RenderScript on the Nexus 5, we observe that it is more energy efficient
  compared to RenderScript.
For the Nexus 7, in some benchmarks RenderScript has better energy utilization while OpenCL is better in others.

Therefore, which implementation to choose to reduce energy consumption is
device- and compute-pattern-dependent.
Compared to the other implementations, RenderScript has lower energy
consumption for many of the benchmarks.
It is also clear that the LCD screen makes a big impact on power usage, in
MRIQ, for example, even though the threaded Java code
  utilizes all the CPU cores, because it is able to complete quicker it uses
less energy overall compared to the serial Java code.

