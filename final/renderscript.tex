
\subsection{RenderScript}
RenderScript was released by Google as an official computing framework in
2011~\cite{RenderScript}. The motivation behind is to provide
performance and portability across SoC architectures for Android devices.

A RenderScript application consists of three parts: (1) Java application host
code written by the developer that runs on Dalvik VM, (2) RenderScript code
written in restricted C99 syntax containing one or more kernels, and (3)
auto-generated Java code that helps the application host code to communicate with
the RenderScript kernel code, allowing functions such as memory binding between the
host program and the kernels.

\begin{figure}
\centering
\includegraphics[scale=0.28]{figs/renderscript-compile.png}
\caption{RenderScript Compilation Flow}
\label{fig:RSCompilation}
\centering
\end{figure}

RenderScript compilation flow is shown in Figure~\ref{fig:RSCompilation}.
First, \fix{llvm-rs-cc} utility is used to compile RenderScript kernels to
LLVM~\cite{LLVM:CGO04} bitcode files. The LLVM IR provide support for a wide range of hardware
devices including CPUs, GPUs and DSPs. 
As part of the RenderScript compilation step, corresponding reflected Java classes
are generated via the  the \fix{llvm-rs-cc} tool.
Then after, the application host code, the reflected Java classes and bitcode
are bundled together into the Android application package (\fix{*.apk} file).
During execution, the RenderScript
runtime invokes \fix{libbcc}, the RenderScript back-end compiler, to translate
bitcode into appropriate machine code.


