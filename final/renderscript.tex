
\subsection{RenderScript}
RenderScript was released by Google as an official computing framework in
2011~\cite{RederScript:url}. The motivation behind is to provide
performance and portability across SoC architectures.

A RenderScript application consists of three parts: (1) Java application host
code written by the developer that runs on Dalvik VM, (2) RenderScript code
written in restricted C99 syntax containing one or more kernels, and (3)
auto-generated Java code that helps application host code to communicate with
RenderScript kernel code, allowing functions such as memory binding between the
host program and the kernels.

RenderScript compilation flow is shown in Figure~\ref{fig:RSCompilation}.
First, \fix{llvm-rs-cc} utility is used to compile RenderScript kernels to
LLVM~\ref{LLVM:url} bitcode files. The LLVM IR provide support for a wide range of hardware
devices including CPUs, GPUs and DSPs. 
As part of the RenderScript compiling step, corresponding reflected Java class
are generated via the  the \fix{llvm-rs-cc} tool.
Then after, the application host code, the reflected Java classes and bitcode
are bundled together into the Android application package (\fix{*.apk} file).
During execution, the RenderScript
runtime invokes \fix{libbcc}, the RenderScript back-end compiler, to translate
bitcode into appropriate machine code.


