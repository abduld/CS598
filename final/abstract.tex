\begin{abstract}

Mobile devices have become ubiquitous with one of their main selling points
	shifting from performance to energy efficiency.
Given that specialized hardware decrease energy
	utilizations for certain workloads, mobile devices have 
	become heterogeneous to compete in the efficient energy market.
But as the number of different heterogeneous devices increases, so does 
	the programming complexity.
Programming languages, such as OpenCL, which expose low level architectural 
	details --- either via the standard or implementation extensions --- 
	results in optimized code for certain architectures and not others.
RenderScript achieves performance portability by both hiding architectural
	details, performing off-line optimizations to an IR, and delaying 
	backend code generation until runtime.
To aid the runtime, the IR is decorated with hints to
	allow the runtime to make intelligent decisions to determine
	where code is to be executed.
We show, by (i) rewriting the Parboil benchmark suite to run on Android,
	and (ii) comparing the performance and energy utilization of Java, threaded Java, C, OpenMP, OpenCL, and RenderScript,
	that RenderScript strikes a balance by
	hiding enough of the architectural details while still achieving
	both good performance and energy utilization.

Our code is publicly available at \url{https://github.com/cmpham/RSBench}.

\end{abstract}
