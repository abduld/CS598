\section{Introduction}

Heterogeneous computing promises to address the rising power dissipation problem
of today's traditional homogeneous multi-core
systems. It provides the ability to integrate a variety of processing elements,
such as large and small general purpose cores, GPUs, DSPs, and custom or
semi-custom hardware into a single system. Applications that can efficiently use
the full range of available hardware reap significant energy
savings over conventional processors by executing portions of the computation on
devices that perform the computational pattern efficiently.
This promise of performance along with power efficiency
has led mobile devices such as smartphones
and tablets, which deal with a variety of applications with limited battery
life, to move towards heterogeneous designs.

However, heterogeneity of hardware resources also has led to a diverse landscape
of different programming models, run-time systems, profiling and debugging tools
for application development. Since low-level performance and energy
optimizations for one device or
programming model often do not work with another programming model or device, 
one gets specialization of code which requires hand tuning by experts for each 
programming model or device, e.g., an optimization for the GPU may degrade performance
on a DSP and vice-versa. This is highly inefficient and
unproductive: we cannot expect applications to use a separate language for each
class of compute unit. If one is to expect applications to use the full range of
available hardware (to maximize performance or energy efficiency or both) then
programming environment has to provide common abstractions to facilitate this.

The industry and the research community have been actively working on abstractions
    to mitigate this problem.
The recent development of RenderScript~\cite{wiki:RenderScript, RenderScript}
provides a framework for running computationally intensive tasks at a high
performance by using a specialized runtime for parallelizing work across all
processors available on the device, such as multi-core CPUs, GPUs, or DSPs.
RenderScript is therefore removing the burden of load balancing and memory
management from the programmer to the run-time, unlike other solutions such as
OpenCL, where the programmer has more control over the execution semantics of
the application (the programmer needs to determine where code is run and
 which part of the memory hierarchy data would reside).  In this fashion,
RenderScript is making the computationally intensive part of the application,
that needs to be accelerated on specialized hardware, performance portable
across the various hardware compute units. Also, since the application is not
dependent on the existence and availability of a specific accelerator, the
application is portable across SoCs with varying combinations of compute units.

While such portability is a noble goal, RenderScript achieves it at the cost of
hiding hardware details from the programmer that are critical to good
performance on these accelerators. For example, in GPUs, the placement of data at
various levels of memory hierarchy is critical for peek performance.  It is this
reason that most programming languages for GPUs allow the programmer fine grained
control over memory management. RenderScript too can use GPUs for acceleration,
but hides the memory hierarchy from the programmer. In the
RenderScript model, application developers only define the part of the
application that needs to be accelerated, and the granularity at which data
needs to be partitioned. The rest of the responsibilities of memory
hierarchy management, work distribution among different compute units, and device selection
is handled by the RenderScript compiler and run-time. 
This raises the question: ``how
effective is the RenderScript compiler and runtime, and is it able to achieve these goals?''

In this report we answer this question 
by performing a comprehensive performance analysis of RenderScript and compare
it against 
serial Java, threaded Java, native C, OpenMP, and OpenCL.
The report is divided as follows: section~\ref{sec:background} introduces 
RenderScript, section~\ref{sec:benchmarks} describes the benchmarks written for
Android and the dataset sizes
used for the analysis,
section~\ref{sec:implementation} gives some internals of how we structured our
implementation of these benchmarks,
section~\ref{sec:analysis} shows results from our analysis which
examine the performance along with processor and energy utilization for the
different implementations,
section~\ref{sec:related} describes related work,
section~\ref{sec:future} details the next logical steps that should be taken to
enhance the benchmark suite and our insights of the results,
finally, section~\ref{sec:conclusion} gives our conclusions.
