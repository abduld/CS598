\section{Introduction}

Heterogeneous computing promises to address the rising power dissipation problem
of today's traditional homogeneous multi-core
systems. It provides the ability to integrate a variety of processing elements,
such as large and small general purpose cores, GPUs, DSPs, and custom or
semi-custom hardware into a single system. Applications that can efficiently use
the full range of available hardware reap significant energy
savings over conventional processors by executing portions of the code on the
device which optimized for it. This promise of performance along with power efficiency
has led mobile devices such as smartphones
and tablets, which deal with a variety of applications with limited battery
life, to move towards heterogeneous designs.

However, heterogeneity of hardware resources also has led to a diverse landscape
of different programming models, run-time systems, profiling and debugging tools
for application development. The differences are so deep that programmers are
often experts on only one class of device, e.g., an expert GPU programmer will
not have much DSP expertise and vice-versa. This is highly inefficient and
unproductive: we cannot expect applications to use a separate language for each
class of compute unit. If we want applications to use the full range of
available hardware to maximize performance or energy efficiency or both, the
programming environment has to provide common abstractions for available
hardware compute units.

The industry and the research community have been trying to solve this problem.
The recent development of RenderScript~\cite{wiki:RenderScript, RenderScript}
provides a framework for running computationally intensive tasks at a high
performance by using a specialized runtime for parallelizing work across all
processors available on the device, such as multi-core CPUs, GPUs, or DSPs.
RenderScript is therefore removing the burden of load balancing and memory
management from the programmer to the run-time, unlike other solutions such as
OpenCL, where the programmer has more control over the execution semantics of
the application ({\em the programmer decides which part of application would run
on which device and using which part of the memory heirarchy}).  In this fashion
RenderScript is making the computationally intensive part of the application,
that needs to be accelerated on specialized hardware, performance portable
across the various hardware compute units. Also, since the application is not
dependent on the existence and availability of a specific accelerator, the
application is portable across SoCs with varying combinations of compute units.

While such portability is a noble goal, RenderScript achieves it at the cost of
hiding hardware details from the programmer that are critical to good
performance on these accelerator. For example, in GPUs, the placement of data at
various levels of memory hierarchy is critical to good performance.  It is this
reason that most programming languages for GPUs, allow the programmer unlimited
control over memory management. RenderScript too can use GPUs for acceleration,
but completely hides the memory management from the programmer. In the
RenderScript model, application developers only define the part of the
application that needs to be accelerated, and the granularity at which data
needs to be partitioned, while the rest of the responsibilities of memory
management and work distribution among different compute units is handled by
RenderScript compiler and run-time.  This raises an important question of ``how
effective is the RenderScript compiler and run-time?'', which we plan to answer
by doing a comprehensive performance analysis of RenderScript.

This section summarizes our progress to date. Table~\ref{table:parboil} shows the
porting status of each version of the benchmarks in the Parboil Benchmark Suite.
In this phase, we also worked on improving the framework, making it easy to collect
data and plot graphs. In parallel, we worked on getting OpenCL to work on our
devices (this was required as Android does not have official support for OpenCL
on Nexus tablets), and started porting the OpenCL version of the benchmarks. 
Further information on each task is described in the subsequent sections.

Our code is publicly available at \url{https://github.com/cmpham/RSBench}.

\begin{table}[h]\small
\centering
\begin{tabular}{ | l | c | c | c | c | c | c |}
    \hline 
    Benchmark & \multicolumn{6}{|c|}{Porting status} \\ \cline{2-7}
                      & C & OMP & J & JT & OCL & RS \\ \hline
    VectorAdd         & C & C   & C    & C      & C      & C \\ \hline
    SGEMM             & C & C   & C    & C      & C      & C \\ \hline
    Stencil           & C & C   & C    & C      & C      & C \\ \hline
    CUTCP             & I & I   & C    & C      & C      & C \\ \hline
    MRI-Q             & I & I   & C    & C      & C      & C \\ \hline
    TPACF             & B & B   & C    & C      & C      & C \\ \hline
    Histogram         & C & B   & C    & C      & C      & C \\ \hline
    BFS               & \multicolumn{6}{|c|}{N/A} \\ \hline
    MRI-G             & \multicolumn{6}{|c|}{N/A} \\ \hline
    SAM               & \multicolumn{6}{|c|}{N/A} \\ \hline
    SPMV              & \multicolumn{6}{|c|}{N/A} \\ \hline
    LBM               & \multicolumn{6}{|c|}{N/A} \\ \hline
    \hline
\end{tabular}
\caption{Parboil Benchmark Porting Status. C : Native C; OMP : Native C with
OpenMP; JT: Threaded Java; OCL : OpenCL;  RS: RenderScript; C: Completed; I: Incomplete; P:
Partial; B : a bug causes the benchmark to crash.}
\label{table:parboil}
\end{table}

