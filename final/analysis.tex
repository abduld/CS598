\section*{Analysis}
\label{sec:analysis}

Unlike programming desktops, where one mainly 
  improves software by increasing either features or performance,
  mobile programmers develop for increase features, performance, and 
  battery life.
In fact, one of the main selling points for hetrogenous
  programming on mobile devices, is the increase in battery life.
Increasingly, hardware vendors, such as Apple or Samsung, sell new
  mobile hardware by advertising long battery life.

In this section we analyze hetrogenous programming on Android battery
  considering not only the performance, but also the power usage.
We conclude the section by discussing the programability of RenderScript
  and compare it against established programming models such as using
  native C, OpenMP, or OpenCL for kernel code.


\begin{table}[h]\small
\centering
\begin{tabular}{ | l | p{2.5cm} | p{1.75cm} | c |}
    \hline 
    Name & CPU & GPU & Memory \\ \hline
    GalaxyNexus & ARMv7, 2 cores, 1200 Mhz, SIMD NEON & PowerVR-SGX 540 & 694Mb \\ \hline
    Nexus5 & Qualcomm Snapdragon S4 Pro 1.5GHz & Adreno 320 400MHz & 2Gb \\ \hline
    Nexus7 & Qualcomm Snapdragon 800 2.26GHz & Adreno 330 450MHz & 2Gb \\ \hline
    Nexus10 & TODO & TODO & TODO \\ \hline
    SM-T900 & QuadCore 1.9GHz Cortex-A15 & Mali T628 MP6 & 3Gb \\ \hline
    \hline
\end{tabular}
\caption{The hardware specifications of the devices we ran the benchmarks on. Two of the devices (Nexus 7 and GalaxyNexus) are mobile phones while the other three are tablets. Only the Nexus 5 and Nexus 7 devices are OpenCL capable.}
\label{table:hardware}
\end{table}

\subsection{Performance}

\begin{figure*}
  \centering
  \begin{subfigure}[b]{\textwidth}
          \centering
          \includegraphics[width=0.4\textwidth]{data/legend.pdf}
  \end{subfigure}

  \begin{subfigure}[b]{0.33\textwidth}
      \centering
      \includegraphics[width=0.9\textwidth]{data/VectorAdd_onecompute_time.pdf}
      \caption{VectorAdd}
  \end{subfigure}
  \begin{subfigure}[b]{0.33\textwidth}
      \centering
      \includegraphics[width=0.9\textwidth]{data/Sgemm_onecompute_time.pdf}
      \caption{Sgemm}
  \end{subfigure}
  \begin{subfigure}[b]{0.33\textwidth}
      \centering
      \includegraphics[width=0.9\textwidth]{data/Mriq_onecompute_time.pdf}
      \caption{MRIQ}
  \end{subfigure}

  \begin{subfigure}[b]{0.33\textwidth}
      \centering
      \includegraphics[width=0.9\textwidth]{data/Tpacf_onecompute_time.pdf}
      \caption{TPACF}
  \end{subfigure}
  \begin{subfigure}[b]{0.33\textwidth}
      \centering
      \includegraphics[width=0.9\textwidth]{data/Histogram_onecompute_time.pdf}
      \caption{Histogram}
  \end{subfigure}
  \begin{subfigure}[b]{0.33\textwidth}
      \centering
      \includegraphics[width=0.9\textwidth]{data/Stencil_onecompute_time.pdf}
      \caption{Stencil}
  \end{subfigure}
  \caption{Runtime across devices where kernel is executed once. The runtimes are normalized to the Java execution time (lower is better). J : Java, JT : JavaThreaded, C : Native C, OMP: OpenMP, OCL : OpenCL, and RS : Renderscript.}
  \label{fig:perfOne}
\end{figure*}

\begin{figure*}

  \centering
  \begin{subfigure}[b]{\textwidth}
          \centering
          \includegraphics[width=0.4\textwidth]{data/legend.pdf}
  \end{subfigure}

  \begin{subfigure}[b]{0.33\textwidth}
      \centering
      \includegraphics[width=0.9\textwidth]{data/VectorAdd_time.pdf}
      \caption{VectorAdd}
  \end{subfigure}
  \begin{subfigure}[b]{0.33\textwidth}
      \centering
      \includegraphics[width=0.9\textwidth]{data/Sgemm_time.pdf}
      \caption{Sgemm}
  \end{subfigure}
  \begin{subfigure}[b]{0.33\textwidth}
      \centering
      \includegraphics[width=0.9\textwidth]{data/Mriq_time.pdf}
      \caption{MRIQ}
  \end{subfigure}

  \begin{subfigure}[b]{0.33\textwidth}
      \centering
      \includegraphics[width=0.9\textwidth]{data/Tpacf_time.pdf}
      \caption{TPACF}
  \end{subfigure}
  \begin{subfigure}[b]{0.33\textwidth}
      \centering
      \includegraphics[width=0.9\textwidth]{data/Histogram_time.pdf}
      \caption{Histogram}
  \end{subfigure}
  \begin{subfigure}[b]{0.33\textwidth}
      \centering
      \includegraphics[width=0.9\textwidth]{data/Stencil_time.pdf}
      \caption{Stencil}
  \end{subfigure}

  \caption{Runtime across devices where kernel is executed multiple times. The runtimes are normalized to the Java execution time (lower is better). J : Java, JT : JavaThreaded, C : Native C, OMP: OpenMP, OCL : OpenCL, and RS : Renderscript.}
  \label{fig:perfMany}
\end{figure*}


The performance measurements are collected by measuring the time
  spent within each section of the code while the device is plugged
  into the development machine.
Each compute part of an implementation is run $5$ times with the minimum
  presented.
We consider two cases --- one where the kernel code is run once (figure~\ref{fig:perfOne}) and therefore
  the overhead (memory, compilation, and initialization) have an impact,
  and one where the kernel is run $100$ times (figure~\ref{fig:perfMany}) (or $5$ for both TPACF and MRIQ)
  and the overhead has little impact.

For each device, the plot show the time to execute sections of the code normalized
  to the Java execution time.
These times correspond to the $x$-axis of the processor utilization times discussed
  in the previous section (e.g. figure~\ref{fig:loadVecAddSgemm}) --- Trepn is not
  running while collecting these timing results.
Not all benchmarks were run on the GalaxyNexus, this is due to the device
  being low end resulting in a long time to execute some of the benchmarks.

In figure~\ref{fig:perfOne} the compute code is only executed once, it is clear that 
  the overhead of RenderScript on the Nexus 10 (and to some extent the SM-T900) device is consistentently high.
We suspect that the Nexus 5 and Nexus 7 are using a more recent version of the RenderScript library compared to the Nexus 10.
As one would expect, a kernel is executed only once is not a good fit to be ofloaded to either RenderScript or OpenCL.
This is due to overhead playing a big roll with overhead time being order of magnitude bigger than the compute time ---
  i.e. the programmer still needs to understand which sections of the program are very hot and could benifit by not being hosted 
  in Java.

In figure~\ref{fig:perfMany}, we look at the performance if memory management is optimized and the kernel code is executed 
  many times.
Code with a high memory to compute ratio, such as VectorAdd (and SGEMM to some extent), do not perform well using either
  RenderScript or OpenCL (this is due to poor occupancy in the OpenCL case).
For code that has irregular accesses or with a low memory to compute ratio, we see RenderScript's compute time to be similar
  to OpenCL, but is better when also considering overhead time.
Both RenderScript and OpenCL outperform the OpenMP implementation in all benchmarks as well.

As expected, the SGEMM OpenMP timing is similar to that of C, confirming our hypothesis that the compiler was not able
  to interpret the OpenMP pragma.
Because of the privatization, which requires an allocation in a thread, the threaded Java implementation performs poorly and is 
  worse than the serial Java implementation.
Consistently, OpenCL results in better speedups on the Nexus 5 versus the Nexus 7 when compared to the onboard CPU. 

The biggest performance gain comes by not using the JVM, however.
Aside from typical JVM overhead, we notice that these kernels are array access extensive.
Since Java's semantics garantee array accesses are within bounds, an overhead is encured.
Java's floating point semantics also do not match modern hardware (which implement the IEE 754 standard),
  this introduces more overhead where the JVM needs to perform extra checks.
These overheads do not manifest themselves in our native implementations.
We also use unsafe casts to reduce the overhead in the native implementations.




\subsection{Power}

Mobile devices employ dynamic voltage frequency scaling (DVFS),
  this results in power draw of the device being goverened by
  the operating frequency of the processor.
A typical model\tt{TODO:CITE Power tutor} of the power draw at time $t$ is

\[
P(t) = GPUVoltage(GPUFreq(t)) + \sum_{i=1}^{N} CPUVoltage_n(CPUFreq_n(t))
\]

with $N$ begin the number of CPU cores, $GPUFreq(t)$ and $CPUFreq(t)$ are the operating frequencies at time $t$, and $GPUVoltage(f)$ and $CPUVoltage(f)$ are the power draws for the processors at the specified voltage.
Other terms, such as GPS, wireless, and other sensors, can be measured or modeled, but for this analysis we turn them off.

The of the issues with using a model are knowing the the power draw at the 
  frequency (which is not specified by the processor's manifacturer) and the method of reading
  the CPU and GPU counters is varies from device to device.
As a result, we use Qualcomm's Trepn tool \tt{TODO: CITE} to read the hardware
  counters.
Trepn, which is limited to Qualcomm based chipsets, reads internal processor
  counters as well as power rail information, both of which are not otherwise
  available programatically.
We set Trepn to read the counters every $100ms$ and measure the load and power
  usage seperatly to decrease the overhead of the profiler.

\subsubsection{Vector Add}

\subsubsection{SGEMM}

\subsubsection{Stencil}

\subsubsection{Histogram}

\subsubsection{CUTCP}



\subsection{RenderScript Programmability}
Pros:

* Allocation is intuitive: Express data and computation simultaneously.

* Flexibility and performability of C?

* Portable; compare to OpenCL

* Debugging: much better with rsDebug. IDE supports showing compile errors.

* Active community: stackoverflow response time is good and the answers were useful.

Cons:

* No intrinsic synchronization within kernels

* Only support two coordinates.

* Overhead of data transfer. Even a kernel is executed in 

* Lack of documentation and standardization. Example: kernel invocation generation.



