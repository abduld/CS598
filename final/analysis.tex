\section{Analysis}
\label{sec:analysis}


\begin{table}
\centering
\begin{tabular}{ | l | p{2.1cm} | p{1.7cm} | c |}
    \hline 
    Name & CPU & GPU & Memory \\ \hline
    \textbf{GalaxyNexus} & ARMv7, 2 cores, 1200 Mhz, SIMD NEON & PowerVR-SGX 540 & 694Mb \\ \hline
    \textbf{Nexus 5} & Qualcomm Snapdragon S4 Pro 1.5GHz & Adreno 320 400MHz & 2Gb \\ \hline
    \textbf{Nexus 7} & Qualcomm Snapdragon 800 2.26GHz & Adreno 330 450MHz & 2Gb \\ \hline
    \textbf{Nexus 10} & DualCore 1.7GHz Cortex-A15 & Mali T604 & 2Gb \\ \hline
    \textbf{SM-T900} & QuadCore 1.9GHz Cortex-A15 & Mali T628 & 3Gb \\ \hline
    \hline
\end{tabular}
\caption{The device hardware specifications used for the analysis.}
\label{table:hardware}
\end{table}

This section evaluates each benchmark on five devices shown
  in Table~\ref{table:hardware}, all running Android $4.4.2$ (KitKat).
These devices capture the low, mid, and high end mobile and tablet
  devices that are currently available in the market.
Only the Nexus 5 and Nexus 7 devices can be modified to install an
  unofficial OpenCL implementation and only those two devices are fully supported
  by Trepn.
Each benchmark contains multiple implementations, which 
  are analyzed to determine trade-offs between
  processor utilization, performance, and energy usage.
We conclude the section by discussing the programability of RenderScript
  and compare it against established programming models such as:
  native C, OpenMP, and OpenCL.

\subsection{Processor Utilization}


\begin{figure*}[htp]
  \centering

  \begin{subfigure}[b]{\textwidth}
          \centering
          \includegraphics[width=0.4\textwidth]{data/load_legend.pdf}
  \end{subfigure}

  \begin{subfigure}[b]{0.85\textwidth}
       \centering
       \includegraphics[width=\textwidth]{data/load_vectoradd_nexus5.pdf}
       \caption{VectorAdd on Nexus 5}\label{fig:Vecadd5}
   \end{subfigure}
  \begin{subfigure}[b]{0.85\textwidth}
       \centering
       \includegraphics[width=\textwidth]{data/load_vectoradd_nexus7.pdf}
       \caption{VectorAdd on Nexus 7}\label{fig:Vecadd7}
   \end{subfigure}

  \begin{subfigure}[b]{0.85\textwidth}
       \centering
       \includegraphics[width=\textwidth]{data/load_sgemm_nexus5.pdf}
       \caption{Sgemm on Nexus 5}\label{fig:Sgemm5}
   \end{subfigure}
  \begin{subfigure}[b]{0.85\textwidth}
       \centering
       \includegraphics[width=\textwidth]{data/load_sgemm_nexus7.pdf}
       \caption{Sgemm on Nexus 7}\label{fig:Sgemm7}
   \end{subfigure}


  \caption{Processor utilization of VectorAdd and SGEMM for both Nexus 5 and Nexus 7. The $x$ axis is time, and the $y$ axis is normalized to the peek utlization for CPUs or GPUs across implementations.}
  \label{fig:loadVecAddSgemm}
\end{figure*}
\FloatBarrier

Mobile devices use dynamic voltage frequency scaling (DVFS)
  to match performance to power utlization.
The frequency is increased for the processors when the load goes over a certain
  threashold.
Both CPUs and GPUs make use of frequency scaling, but unlike CPUs (which typically
  have vary fine frequencies --- operating at around 10 different frequencies), GPUs
  have coarse grained frequencies (operating at only 4 different frequencies).  

Processor utlization is measured by multiplying the $frequency$ and the $load$ information collected via Qualcomm's Trepn tool.
The performance times are measured with the system connected to a
  power source, which avoids the device going to sleep (it does add the
  overhead of the device comunicating debug information with the development
  machine).
Since Trepn has a $100ms$ measure granularity, the kernel code is run $100$ times
  ($5$ times for MRIQ and TPACF), this allows us to visualize the trend in resource
  utlization.
The information gives us insight as to what parts of the code are active in each
	section of the code, how that impacts performance and battery.


Note that Java UI thread is running independent of the computation as well as Trepn which
  have a high overhead ($10\%$ on average).
These background processes are echoed in
  the plots --- a single threaded Java code, for example, should not utilize more than one core, but our plots show that more than one core is active.


VectorAdd has a very high memory to compute ratio, therefore the processor utlization
  plot~(figure \ref{fig:loadVecAddSgemm}) shows that not all CPUs 
  are fully utlized during the compute phase.
For the Nexus 5 device, while the graphs show a GPU is being used during the OpenMP,
  Renderscript, and Java, this is an error in the measurement --- we believe this is 
  due to the UI thread utilizing the GPU or some other interferences.
As expected, the GPU is not being utilized on the Nexus 7 except for OpenCL.

As discussed in the benchmark section, the Android GCC compiler was not able 
  to interpret the OpenMP pragma code and therefore SGEMM runs in a single thread
  on both the Nexus 5 and Nexus 7 (figure~\ref{fig:loadVecAddSgemm}).
While OpenCL does utlize the GPU, because of the size of the matrix, there GPU has
  lower occupancy and therefore does not achieve peek performance.
Both the RenderScript and ThreadedJava code make full utlization of all the cores.

MRIQ is embarisingly parallel and we can see full utlization for both the CPUs and GPUs in figure~\ref{fig:loadMRIQTpacf}.
Abrupt dips in the plots show positions where a kernel launch occurs.
It is also interesting to note that on the Nexus 5 the CPU is not fully utilized for both ThreadedJava and Renderscript.
This is due to either the load being low for a high frequency or the frequency being 
  choosen to be low.

\begin{figure*}[t]
  \centering

  \begin{subfigure}[b]{\textwidth}
          \centering
          \includegraphics[width=0.4\textwidth]{data/load_legend.pdf}
  \end{subfigure}

  \begin{subfigure}[b]{0.95\textwidth}
      \centering
      \includegraphics[width=\textwidth]{data/load_mriq_nexus5.pdf}
      \caption{MRIQ on Nexus 5}
      \label{fig:MRIQ5}
  \end{subfigure}
  \begin{subfigure}[b]{0.95\textwidth}
      \centering
      \includegraphics[width=\textwidth]{data/load_mriq_nexus7.pdf}
      \caption{MRIQ on Nexus 7}
      \label{fig:MRIQ7}
  \end{subfigure}

  \begin{subfigure}[b]{0.95\textwidth}
      \centering
      \includegraphics[width=\textwidth]{data/load_tpacf_nexus5.pdf}
      \caption{TPACF on Nexus 5}
      \label{fig:TPACF5}
  \end{subfigure}
  \begin{subfigure}[b]{0.95\textwidth}
      \centering
      \includegraphics[width=\textwidth]{data/load_tpacf_nexus7.pdf}
      \caption{TPACF on Nexus 7}
      \label{fig:TPACF7}
  \end{subfigure}

  \caption{Processor utilization of MRIQ and TPACF for both Nexus 5 and Nexus 7. The $x$ axis is time, and the $y$ axis is normalized to the peek utlization for CPUs or GPUs across implementations.}
  \label{fig:loadMRIQTpacf}
\end{figure*}
\FloatBarrier

The TPACF compute codes is divided into two parts.
The first is serial computation that, for the sake of code reuse, is done on the 
  Java side using a single thread.
Once that complets, we then execute the code on different datasets in parallel.
The utlization plots (figure~\ref{fig:loadMRIQTpacf}) show this behavior.  

For Histogram (figure~\ref{fig:loadHistogramStencil}) we see that memory starts
  to play a major role in processor utlization.
The GPU is being utlized in the memory copy as it is in the computation, for example.

For Stencil (figure~\ref{fig:loadHistogramStencil}) we see that RenderScript does
  not fully utilize all CPU cores.
This is because the stencil kernel is memory access bound, performing around 10 flops 
  (ignoring index calculations).
We again see that memory copy (\fix{SetupArgs} for RenderScript and
  \fix{CopyTo} for OpenCL) result in the a substantial amound of resources being
  wasted.

\begin{figure*}[t]
  \centering

  \begin{subfigure}[b]{\textwidth}
          \centering
          \includegraphics[width=0.4\textwidth]{data/load_legend.pdf}
  \end{subfigure}

  \begin{subfigure}[b]{0.9\textwidth}
      \centering
      \includegraphics[width=\textwidth]{data/load_histogram_nexus5.pdf}
      \caption{Histogram on Nexus 5}
      \label{fig:Histogram5}
  \end{subfigure}
  \begin{subfigure}[b]{0.9\textwidth}
      \centering
      \includegraphics[width=\textwidth]{data/load_stencil_nexus5.pdf}
      \caption{Stencil on Nexus 5}
      \label{fig:Stencil5}
  \end{subfigure}

  \caption{Processor utilization of Histogram and Stencil for both Nexus 5. The $x$ axis is time, and the $y$ axis is normalized to the peek utlization for CPUs or GPUs across implementations.}
  \label{fig:loadHistogramStencil}
\end{figure*}
\FloatBarrier

Throughout the benchmarks we see that RenderScript
	does not utalize the GPU.
This is due to RenderScript requiring full
	IEEE 754-2008 compliance, which the GPU is not able to provide.
Inserting the \fix{\#pragma rs\_fp\_imprecise} pragma into the
	RenderScript kernel allows for relaxed IEEE compliance and 
	should allow for GPU acceleration.
Similar options are available in OpenCL and C, but since
	this option is not available in Java we chose to disable it. 
Further investigations would require us to analyze the results with relatex precision.


\subsection{Performance}

\begin{figure*}
  \centering
  \begin{subfigure}[b]{\textwidth}
          \centering
          \includegraphics[width=0.4\textwidth]{data/legend.pdf}
  \end{subfigure}

  \begin{subfigure}[b]{0.33\textwidth}
      \centering
      \includegraphics[width=0.9\textwidth]{data/VectorAdd_onecompute_time.pdf}
      \caption{VectorAdd}
  \end{subfigure}
  \begin{subfigure}[b]{0.33\textwidth}
      \centering
      \includegraphics[width=0.9\textwidth]{data/Sgemm_onecompute_time.pdf}
      \caption{Sgemm}
  \end{subfigure}
  \begin{subfigure}[b]{0.33\textwidth}
      \centering
      \includegraphics[width=0.9\textwidth]{data/Mriq_onecompute_time.pdf}
      \caption{MRIQ}
  \end{subfigure}

  \begin{subfigure}[b]{0.33\textwidth}
      \centering
      \includegraphics[width=0.9\textwidth]{data/Tpacf_onecompute_time.pdf}
      \caption{TPACF}
  \end{subfigure}
  \begin{subfigure}[b]{0.33\textwidth}
      \centering
      \includegraphics[width=0.9\textwidth]{data/Histogram_onecompute_time.pdf}
      \caption{Histogram}
  \end{subfigure}
  \begin{subfigure}[b]{0.33\textwidth}
      \centering
      \includegraphics[width=0.9\textwidth]{data/Stencil_onecompute_time.pdf}
      \caption{Stencil}
  \end{subfigure}
  \caption{Runtime across devices where kernel is executed once. The runtimes are normalized to the Java execution time (lower is better). J : Java, JT : JavaThreaded, C : Native C, OMP: OpenMP, OCL : OpenCL, and RS : Renderscript.}
  \label{fig:perfOne}
\end{figure*}

\begin{figure*}

  \centering
  \begin{subfigure}[b]{\textwidth}
          \centering
          \includegraphics[width=0.4\textwidth]{data/legend.pdf}
  \end{subfigure}

  \begin{subfigure}[b]{0.33\textwidth}
      \centering
      \includegraphics[width=0.9\textwidth]{data/VectorAdd_time.pdf}
      \caption{VectorAdd}
  \end{subfigure}
  \begin{subfigure}[b]{0.33\textwidth}
      \centering
      \includegraphics[width=0.9\textwidth]{data/Sgemm_time.pdf}
      \caption{Sgemm}
  \end{subfigure}
  \begin{subfigure}[b]{0.33\textwidth}
      \centering
      \includegraphics[width=0.9\textwidth]{data/Mriq_time.pdf}
      \caption{MRIQ}
  \end{subfigure}

  \begin{subfigure}[b]{0.33\textwidth}
      \centering
      \includegraphics[width=0.9\textwidth]{data/Tpacf_time.pdf}
      \caption{TPACF}
  \end{subfigure}
  \begin{subfigure}[b]{0.33\textwidth}
      \centering
      \includegraphics[width=0.9\textwidth]{data/Histogram_time.pdf}
      \caption{Histogram}
  \end{subfigure}
  \begin{subfigure}[b]{0.33\textwidth}
      \centering
      \includegraphics[width=0.9\textwidth]{data/Stencil_time.pdf}
      \caption{Stencil}
  \end{subfigure}

  \caption{Runtime across devices where kernel is executed multiple times. The runtimes are normalized to the Java execution time (lower is better). J : Java, JT : JavaThreaded, C : Native C, OMP: OpenMP, OCL : OpenCL, and RS : Renderscript.}
  \label{fig:perfMany}
\end{figure*}


The performance measurements are collected by measuring the time
  spent within each section of the code while the device is plugged
  into the development machine.
Each compute part of an implementation is run $5$ times with the minimum
  presented.
We consider two cases --- one where the kernel code is run once (figure~\ref{fig:perfOne}) and therefore
  the overhead (memory, compilation, and initialization) have an impact,
  and one where the kernel is run $100$ times (figure~\ref{fig:perfMany}) (or $5$ for both TPACF and MRIQ)
  and the overhead has little impact.

For each device, the plot show the time to execute sections of the code normalized
  to the Java execution time.
These times correspond to the $x$-axis of the processor utilization times discussed
  in the previous section (e.g. figure~\ref{fig:loadVecAddSgemm}) --- Trepn is not
  running while collecting these timing results.
Not all benchmarks were run on the GalaxyNexus, this is due to the device
  being low end resulting in a long time to execute some of the benchmarks.

In figure~\ref{fig:perfOne} the compute code is only executed once, it is clear that 
  the overhead of RenderScript on the Nexus 10 (and to some extent the SM-T900) device is consistentently high.
We suspect that the Nexus 5 and Nexus 7 are using a more recent version of the RenderScript library compared to the Nexus 10.
As one would expect, a kernel is executed only once is not a good fit to be ofloaded to either RenderScript or OpenCL.
This is due to overhead playing a big roll with overhead time being order of magnitude bigger than the compute time ---
  i.e. the programmer still needs to understand which sections of the program are very hot and could benifit by not being hosted 
  in Java.

In figure~\ref{fig:perfMany}, we look at the performance if memory management is optimized and the kernel code is executed 
  many times.
Code with a high memory to compute ratio, such as VectorAdd (and SGEMM to some extent), do not perform well using either
  RenderScript or OpenCL (this is due to poor occupancy in the OpenCL case).
For code that has irregular accesses or with a low memory to compute ratio, we see RenderScript's compute time to be similar
  to OpenCL, but is better when also considering overhead time.
Both RenderScript and OpenCL outperform the OpenMP implementation in all benchmarks as well.

As expected, the SGEMM OpenMP timing is similar to that of C, confirming our hypothesis that the compiler was not able
  to interpret the OpenMP pragma.
Because of the privatization, which requires an allocation in a thread, the threaded Java implementation performs poorly and is 
  worse than the serial Java implementation.
Consistently, OpenCL results in better speedups on the Nexus 5 versus the Nexus 7 when compared to the onboard CPU. 

The biggest performance gain comes by not using the JVM, however.
Aside from typical JVM overhead, we notice that these kernels are array access extensive.
Since Java's semantics garantee array accesses are within bounds, an overhead is encured.
Java's floating point semantics also do not match modern hardware (which implement the IEE 754 standard),
  this introduces more overhead where the JVM needs to perform extra checks.
These overheads do not manifest themselves in our native implementations.
We also use unsafe casts to reduce the overhead in the native implementations.


\subsection{Power}

Mobile devices employ dynamic voltage frequency scaling (DVFS),
  this results in power draw of the device being goverened by
  the operating frequency of the processor.
A typical model\tt{TODO:CITE Power tutor} of the power draw at time $t$ is

\[
P(t) = GPUVoltage(GPUFreq(t)) + \sum_{i=1}^{N} CPUVoltage_n(CPUFreq_n(t))
\]

with $N$ begin the number of CPU cores, $GPUFreq(t)$ and $CPUFreq(t)$ are the operating frequencies at time $t$, and $GPUVoltage(f)$ and $CPUVoltage(f)$ are the power draws for the processors at the specified voltage.
Other terms, such as GPS, wireless, and other sensors, can be measured or modeled, but for this analysis we turn them off.

The of the issues with using a model are knowing the the power draw at the 
  frequency (which is not specified by the processor's manifacturer) and the method of reading
  the CPU and GPU counters is varies from device to device.
As a result, we use Qualcomm's Trepn tool \tt{TODO: CITE} to read the hardware
  counters.
Trepn, which is limited to Qualcomm based chipsets, reads internal processor
  counters as well as power rail information, both of which are not otherwise
  available programatically.
We set Trepn to read the counters every $100ms$ and measure the load and power
  usage seperatly to decrease the overhead of the profiler.

\subsubsection{Vector Add}

\subsubsection{SGEMM}

\subsubsection{Stencil}

\subsubsection{Histogram}

\subsubsection{CUTCP}



\subsection{Input/Output}

Throught the analysis, we have ignored IO times which is infact the biggest performance bottleneck in these benchmarks.
Unlike desktops which have a harddrive read speed of 80-100MB/s, Android devices use flash which have a 7-25MB/s read speed.
In reality, however, computational data for mobile devices are not stored on disk and are usually either streamed from the cloud or captured from onboard sensors and therefore are available in RAM.
To keep with the typical usage, and to not skew the plots, we therefore explicitly removed the IO performance times from our graphs.


\subsection{RenderScript Programmability}
Pros:

* Allocation is intuitive: Express data and computation simultaneously.

* Flexibility and performability of C?

* Portable; compare to OpenCL

* Debugging: much better with rsDebug. IDE supports showing compile errors.

* Active community: stackoverflow response time is good and the answers were useful.

Cons:

* No intrinsic synchronization within kernels

* Only support two coordinates.

* Overhead of data transfer. Even a kernel is executed in 

* Lack of documentation and standardization. Example: kernel invocation generation.



