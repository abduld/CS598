\section{Methodology}

There is one major change in our strategy to evaluate RenderScript from our last
progress report. In our last report we mentioned that because of the ongoing competition between
Google's RenderScript and Khronos Group's OpenCL, Google has decided not to ship
GPU-CPU RenderScript drivers in their Nexus phones/tablets firmware. They never
officially supported the drivers for use by developers, but the Android 4.2 was
found to have unofficial OpenCL drivers. It was predicted that these drivers
would be removed from future versions and as expected Google removed these drivers from
Android 4.3 onwards. As a result we decided to focus on evaluating RenderScript
implementation against Single threaded and multi-threaded Java implementations. 

However, we have been able to manually install OpenCL drivers in Android 4.4.2
running on our Nexus 7 device and we would now be able to evaluate
RenderScript against OpenCL.
This requires one to hack the device and can be done for the Nexus 5 device, but
  we have not done so already.


\subsection{Performance Analysis}

We use the timer discussed in the previous section as our means to measure performance.
For each implementation, we run the compute section multiple times taking the 
  minimum time across runs.
We carefully place hints to Java so that it performs the garbage collection before
  it enters the compute part of code --- this results in the Java garbage collector not being invoked during the compute part of our code by chance.

While we did not do so in our current measures, we plan on making sure
   we exit all applications before running our benchmark suite.
While we do not think that this to be a major factor, we will err on the side of caution.

\subsection{Power Analysis}

No progress has been made in trying to figure out the APIs needed to be called
  to perform the power analysis.
We may resort to some tools to run the power analysis, but given that the majority
  of the runtime is devoted to I/O, we need to understand how to do power
  analysis without accounting for the I/O time.

