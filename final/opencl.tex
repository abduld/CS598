
\subsection{OpenCL}

OpenCL was initiated by Apple Inc. as a vendor neutral alternative
to CUDA and is now managed by the Khronos Group~\cite{OpenCL_kh}.
An OpenCL application is composed of two parts: an OpenCL host program and a
set of one or more kernels. The kernels, written in restricted C99 syntax,
specify functions that are to be executed in data parallel fashion. The OpenCL
host program identifies the devices on which the OpenCL kernel would be
executed, sets up the environment, allocate memory on the device, copy data
into the device memory and enqueue the kernel execution on the device.


To allow execution on different devices, OpenCL programs represent the kernel
	code as a C string.
The OpenCL runtime contains a compiler which then compiles the C code at runtime.
This extra compilation step results in large overhead. Future OpenCL implementations
will have the ability to perform some off line compilation to a virtual instruction
set~\cite{SPIR}.

In the ideal case, OpenCL allows application developers to write
data parallel computation intensive parts of their application for an abstract
hardware model, without using low level hardware specific function calls.
In reality, developers need to target specific hardware to get tangential
performance benefit. Furthermore, developers often use hardware specific 
parameters such as the size of OpenCL work-group, shared memory size either 
to obtain better performance on specific hardware or because the OpenCL API
is verbose that it is inconvenient to extract that data pragmatically.
This harms the portability of OpenCL application kernel code.

