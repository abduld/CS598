
\subsection{OpenCL}
OpenCL was initiated by Apple Inc. and is managed by the Khronos
Group~\cite{Khronos:url}. It allows the application developers to write
data parallel computation intensive parts of their application for an abstract
hardware model, without using low level hardware specific function calls.

An OpenCL application is composed of two parts: an OpenCL host program and a
set of one or more kernels. The kernels, written in restricted C99 syntax,
specify functions that are to be executed in data parallel fashion. The OpenCL
host program identifies the device on which the OpenCL kernel would be
executed, sets up the environment, allocate memory on the device, copy data
into the device memory and enqueue the kernel execution on the device.

In the Android world, where all application execute on Dalvik virtual machine,
an application using OpenCL has a third component as well. This is the Java
host code which would initialize the application, read in the inputs and use
JNI calls to invoke OpenCL host code.

OpenCL is designed to be portable across a wide range of devices. However,
developers often use hardware specific parameters such as the size of OpenCL
work-group, shared memory size to obtain better performance on specific
hardware. This harms the portability of OpenCL application kernel code.

