\subsection{RenderScript Programmability}
This section presents our analysis of RenderScript's programmability --- how
comfortable we, as programmers, feel when working with RenderScript. Since this
analysis is primarily based on our experience working with RenderScript
throughout this semester, it is subjective. Furthermore, we are fully aware of
that some of the limitations presented in this section are due to the fact that
RenderScript is a new framework being rapidly developed and deployed.
Therefore, we expect that RenderScript will keep evolving to close most of the
identified limitations.

Pros:

* Allocation is intuitive: Express data and computation simultaneously.

* Flexibility and performability of C?

* Portable; compare to OpenCL

* Debugging: much better with rsDebug. IDE supports showing compile errors.

* Active community: stackoverflow response time is good and the answers were useful.

Cons:

* No intrinsic synchronization within kernels

* Only support two coordinates.

* Overhead of data transfer. Even a kernel is executed in 

* Lack of documentation and standardization. Example: kernel invocation generation.


