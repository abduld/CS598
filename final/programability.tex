\subsection{RenderScript Programmability}

This section presents our analysis of RenderScript's programmability.
This analysis is based on our experience working with RenderScript, and is 
  is highly subjective.
Some of the limitations presented are due to RenderScript being a less mature
  framework (compared to CUDA and OpenCL) and is actively being developed.
Therefore, we expect that RenderScript will evolve to fix many of the
identified limitations and fix the unnecessary idiosyncratic restrictions we encountered.

\subsubsection{Performance Portability}

The main objective of RenderScript is to give acceptable compute performance
 (comparable to OpenCL) on a variety of hardware.
The runtime behavior of our RenderScript implementations is consistent across
the devices we tested on.
Reproducibility of bugs is another important factor, we
always got the same runtime errors, e.g., segmentation fault or out-of-memory
error, regardless of the devices that the code is running on.

RenderScript does not expose any device specific features --- it is not even 
 possible to choose which device RenderScript runs on.
Regardless, RenderScript was able to achieve good performance and work unchanged 
 across devices.
OpenCL, on the other hand, required us to read some device specific parameters
 and change some of the run parameters, such as work-group sizes, in order to avoid 
 runtime errors.

\subsubsection{Debug Support}

Since RenderScript is natively supported in Android,
it offers a set of convenient built-in debugging
facilities, such as the \fix{rsDebug} functions, detailed runtime exceptions,
and IDE-navigable compilation errors.
The \fix{rsDebug} function forces RenderScript to execute on the CPU and interacts 
with the eclipse platform to offer a continent way to present printed values from within the kernel
(OpenCL and CUDA offer similar facilities, but they are less convenient to use).
For programmers accustomed to \fix{printf} debugging style in C/C++, this interface provides
a familiar debugging model. For developers expecting a breakpoint debugging workflow, the eclipse
environment has no facilities for that currently.


\subsubsection{Memory Operations}

The \fix{Allocation} interface is intuitive to express data and
execution parallelism. As described in section~\ref{sec:implementationRS}, this
\fix{Allocation} interface gives the programmer the ability to express parallelism
granularities, i.e., via packing output and/or input into \fix{Element} objects.
The interface is intuitive and is similar to the already familiar OpenCL memory interface.
Within the kernel, utility functions (\fix{rsGetElementAt\_*}) make indexing
 into a multi-dimensional arrays platform agnostic
 (e.g. the strides should not be assumed to be the same across architectures).
The helper functions also allow one to not perform complex index arithmetic, mapping
 a set of coordinates to access a $1$D array.


\subsubsection{Familiar Language}

RenderScript kernel is C99 based, and therefore does not require any learning of syntax.
Classifying that as a features is slightly biased towards programmers already familiar with
 C programming.
For many programmers, a transition from programming in Java to programming in C might not be simple
 --- especially if the two languages are within the same project.
Similarly, for C programmers, having a language that is C like but employing different semantics
  might be confusing.


\subsubsection{Multi-Dimensional Parallelism}


Unlike the familiar CUDA and OpenCL model where threads are partitioned into blocks which are
  then partitioned into a grid, RenderScript has only one level partitioning.
A RenderScript kernel allows an \fix{Allocation} to specify \fix{X}, \fix{Y}, and \fix{Z}
dimensions, and the workload would be distributed across hardware units using
these dimensions.
We found that the current implementation lacks support for launching a $3$D kernels.

\subsubsection{Lack of Synchronization Intrinsics}

Because RenderScript does not allow one to group threads into blocks, there is only a global synchronization operation
 (using \fix{syncAll})  which is called from within the Java code.
Coming from a GPU programming background, this means no support for \fix{\_\_syncthread} as in CUDA or the OpenCL
equivalent \fix{barrier} function.
The RenderScript kernels therefore cannot perform fine grained synchronizations and share data between a group of threads.
For some benchmarks, this model is not convenient.
In the \fix{CUTCP} benchmark, for example, we had to rethink how the algorithm operates to port 
  the OpenCL implementation to RenderScript.


\subsubsection{Non-Unified Memory}

Before RenderScript execution starts, all
heap data to be used by the kernel has to copied into \fix{Allocation} buffers.
The buffer has to be copied back to make it available to Java.
Safe casting between Java and C requires some checking --- since Java does not 
  conform to the same IEEE standard that RenderScript conforms to.
But in the case of unsafe copies, these can be avoided when the kernel is
  executing in the same coherence domain.
The current implementation of RenderScript does not perform these optimizations.
Furthermore, the runtime, via function overloading, should be able to hide 
  explicit buffer creation and copies, by detecting the type passed into the function
  and converting to a \fix{Allocation} buffer if what is passed is a Java array.

\subsubsection{Lack of a Standardization}

RenderScript is similar to CUDA in this respect --- the reference implementation
  is the standard.
Yet unlike CUDA, RenderScript's documentation tend to be incomplete and insufficient.
Atomic instructions such as \fix{rsAtomicInc}, for example, are claimed to be support 
(in both the documentation and the header files) several data types, but a 
runtime exception is raises when use with types other than \fix{int32\_t}.
The lack of specification results in some errors being unintuitive --- one has to 
  read the runtime to determine what the error refers to and what caused it.
It also means that it is unlikely that RenderScript would be 
  ported to desktop platforms.


\subsubsection{Yet Another Parallel Framework}

Aside from having full control over the language~\cite{googlelockin}, it is not clear why
 Google did not adopt OpenCL or OpenCL's SPIR layer.
As can be seen, the RenderScript language borrows many elements from the CUDA/OpenCL 
 programming model, with some tooling support.
Many of the ``features'' of RenderScript can be implementing via a library that 
 interacts with OpenCL and there is no need for a new language.
Futhermore, many hadware vendors already support OpenCL~\cite{OpenCL:Android}.

One reason for a new language could be that the compiler and runtime can prevent compute code that run on the GPU from
 possibly crashing the driver (by using too much resources or using unsafe memory accesses),
 since crashing the GPU driver on mobile devices is equivalent to crashing the device.
Yet, again, this feature can be implemented by enhancing the OpenCL compiler and runtime to detect
 and raise an exception if illegal memory accesses occur.
It is therefore unclear why a new programming language was necessary.
