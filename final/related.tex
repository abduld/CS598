\section{Related Work}
\label{sec:related}

Related work can be divided into $3$ categories : works that
explore performance portability, works that examine the
programming models and how to abstract some of the low level details them, and,
finally, works that examine computational patterns on mobile platforms.

\subsection{Performance Portability}

In term of programming model, RenderScript is similar to OpenCL~\cite{OpenCL_kh}
and CUDA~\cite{CUDA:Programming-Guide}. While CUDA is restricted to NVIDIA GPU
devices, OpenCL is vendor neutral. OpenCL performance and performance
portability has been extensively evaluated.  Most evaluations compare OpenCL
against CUDA on GPUs~\cite{fang2011comprehensive, weber2011comparing,
van2011correlating, vassilev2010comparison, amorim2009comparing,
karimi2010performance, komatsu2010evaluating}.  Since OpenCL and CUDA have a
similar platform, memory, and programming model on discrete GPUs, a one-to-one
analysis is fair.  Using benchmarks, such as Parboil, ~\cite{weber2011comparing,
van2011correlating, vassilev2010comparison, amorim2009comparing} show that on
NVIDIA GPUs, CUDA achieves better performance than OpenCL.  The studies also
show that OpenCL provides a sufficient interface for developers to express more
architectural details to improve the performance of their applications. 

In ~\cite{komatsu2010evaluating,fang2011comprehensive,dolbeau2013one} the
authors show that most OpenCL kernels can obtain comparable performance with
CUDA kernels when properly optimized.  Furthermore, the authors show that OpenCL
achieves stable performance across the tested platforms. The studies illustrate,
however, that cases exist where OpenCL does not handle architectural specifics
well, such as memory layout and number of processing cores. In order to improve
the portability of applications, recent OpenCL iterations have added the option
to delegate the workgroup size to the runtime.  We are not aware of any study
that evaluates the optimality of this feature.

% Compared with OpenCL, RenderScript's programming model is more restrictive, in
% the sense that it does not let developers to control the execution scheduling,
% i.e., developers do not know whether a particular region of code is going to be
% executed in a CPU or in a GPU at runtime. RenderScript also limits developers
% from expressing architectural specifics, such as the number of processing cores,
% local memory size. This study will evaluate whether hiding architectural details
% results in RenderScript incurring performance loss compared to OpenCL.

\subsection{Programming Model Abstractions}


There are many recent proposals aiming at abstracting parallel software
development to either make it easier to develop, less error prone, or to have
better utilization heterogeneous cores on mobile devices.
Aparapi~\cite{Aparapi_web} allows one to off load computation to OpenCL by
inserting Java source code annotations around the compute kernels. In
~\cite{6704508} the authors extend the Aparapi model, introducing
Android-Aparapi, and targets Android devices.

Intel~\cite{barik2014efficient} has extended the Thread Building Blocks
(TBB)~\cite{reinders2007intel} model to run on heterogeneous devices. In this
model, one encodes parallelism as a DAG of tasks.  The tasks are then schedule
to run in parallel on available hardware (maintaining the dependencies that the
programmer encoded).  Qualcomm MARE~\cite{MARE_qc} is similar to Intel's TBB
extension.  In~\cite{khaldi2013task}, the authors summarize the programming
model of common task based parallelism implementations.

Other work examined whether a source-to-source translator is practical to port
OpenCL code to RenderScript.  In ~\cite{yang2012o2render}, the authors presents
such a tool along with challenges of this process: the most notable is
differences in the execution models of the two.  More recently, in
~\cite{6800300} the authors develop a domain specific language that allows one
to develop image processing kernels, but target C, CUDA, OpenCL, or
RenderScript. In ~\cite{alejandro2014performance}, the authors propose the
Paralldroid language which allows one to target C and RenderScript using
programmer's annotation of Java source code. The results show that the
auto-generated RenderScript code often achieves higher performance than the
auto-generated C code. 

\subsection{Computation on Mobile Devices}

Studies examining the computation behavior on mobile devices are scarce.  In
\cite{kemp2013using} the authors compare RenderScript, Remote CUDA (RCUDA), and
C for image processing applications. The results show that on an NVIDIA Tegra 3
GPU RenderScript outperforms both the RCUDA and C. Furthermore, this performance
gap increases as the size of the datasize increases.

However, we are not aware of any work that provides a systematic evaluation of
RenderScript's performance and performance portability.  The only available tool
that does some comparison is the CompuBench mobile for
RenderScript~\cite{compuBenchMobile} commercial benchmark application.  This
application is targeted at comparing RenderScript performance across devices,
rather than understanding the results, and the benchmarks used (around 4
applications) are all image processing based.


