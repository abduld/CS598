\section{Related Work}

In term of programming model, RenderScript is similar to OpenCL~\cite{OpenCL}
and CUDA. While CUDA is designed specifically for NVIDIA GPU devices, OpenCL's
goal is similar to RenderScript, which is aiming at simplifying cross-platform
parallel programming for heterogeneous systems.  In fact, Google had an option
to adopt OpenCL, since some android hardware already has OpenCL
SDKs~\cite{OpenCL:Android}, but they opted to create RenderScript to.  Google
justified the choice by arguing that they not only required performance
portability and development efficiency but also a more intuitive programming and
distribution model.  This decision caused some frustration from the OpenCL
community \cite{androidblockopenCL} and some hardware
vendors~\cite{googlelockin} who made big investments on OpenCL.

Since being introduced in 2008, OpenCL performance and performance portability
has been extensively evaluated.  The most common of such evaluations is OpenCL's
performance against CUDA on GPUs~\cite{fang2011comprehensive,
weber2011comparing, van2011correlating, vassilev2010comparison,
amorim2009comparing, karimi2010performance, komatsu2010evaluating}.  Since on
GPUs OpenCL and CUDA have a similar platform, memory, and programming model, a
one-to-one analysis is possible.  Most studies~\cite{weber2011comparing,
van2011correlating, vassilev2010comparison, amorim2009comparing}  from wide
array of domains show that CUDA usually achieve better performance (on NVIDIA
GPUs) than OpenCL.  Another consensus among these studies is that OpenCL
provides a sufficient interface for developers to express more architectural
details to improve the performance of their applications.  For example,
studies~\cite{komatsu2010evaluating} and \cite{fang2011comprehensive} show that
most OpenCL kernels can obtain comparable performance with CUDA kernels when
properly optimized.  In~\cite{shen2012performance}, the author compares OpenCL
and OpenMP in the context of application performance on multi-core CPUs using
the Rodinia benchmark suite~\cite{che2009rodinia}.  From the study, the OpenMP
implementations generally outperforms the OpenCL ones.  Based on this result,
the authors picked three OpenCL worse-performed applications, compared the
performance against OpenMP, and performed manual performance tuning.  Their
result show that tuned OpenCL applications outperformed the OpenMP in majority
of the test cases.

According to \cite{komatsu2010evaluating} and \cite{dolbeau2013one}, OpenCL
achieves fairly stable performance across tested platforms. However, both
studies also illustrate some cases, in which OpenCL does not handle
architectural specifics well, such as memory layout and number of processing
cores. In order to improve the portability of applications, recent OpenCL
versions has an option to let the runtime decides the group size, i.e., the
number of concurrent threads, or wrap size in CUDA's term. However, we are not
aware of any study that evaluates the optimality of this feature.

Compared with OpenCL, RenderScript's programming model is more restrictive, in
the sense that it does not let developers to control the execution scheduling,
i.e., developers do not know whether a particular region of code is going to be
executed in a CPU or in a GPU at runtime. RenderScript also limits developers
from expressing architectural specifics, such as the number of processing cores,
local memory size. This study will evaluate whether hiding architectural details
results in RenderScript incurring performance loss compared to OpenCL.

Several studies have focused on further simplifying the development of
RenderScript code. In order to quickly reuse OpenCL legacy code in Android
environments, Yang et al.  \cite{yang2012o2render} presents a source-to-source
translator from OpenCL to RenderScript. The authors present several challenges
of this process, the most notable one is the differences in the execution models
of the two standards. More recently, Acosta et al.
\cite{alejandro2014performance} proposes a parallel development framework,
called Paralldroid, for improving the parallelism of programs running in the
Android platform. The framework automatically generates parallel code in C and
RenderScript based on programmers' annotation in Java code. The results show
that the auto-generated RenderScript code often achieves higher performance than
the auto-generated Native C code.

In term of benchmarking RenderScript, study \cite{kemp2013using} compares three
different programming models, namely RenderScript, Remote CUDA (RCUDA), and C++,
of an image processing library in Android platform. The results on a Tegra 3
quad core device evidently show that RenderScript outperforms both the RCUDA and
C++ implementations. Furthermore, this performance gap increases as the size of
the input increases.  However, we are not aware of any work that provides a
systematic evaluation of RenderScript's performance and performance portability.
The only available tool that we might be able to leverage is CompuBench mobile
for RenderScript~\cite{compuBenchMobile}.  But since this benchmark is a
commercial product and does not offer source, we would not be able to perform
thorough analysis using this it.  So in this regard, we will be providing the
first open source benchmark for RenderScript that can be evaluated against
different language paradigms and hadware targets.

