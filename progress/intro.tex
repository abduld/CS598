
\section*{Progress Summary}

This section summarizes our progress to date. Further information on each task
is described in the subsequent sections.

We have successfully setup our development infrastructure, which allows us to
efficiently collaborate and develop on both the emulator and hardware devices.
phones.  Due to the diversity of the Android versions (4.2, 4.3, and 4.4), the
SDKs (API-17, API-18, and API-19), and the hardware configurations of different
devices, we had a chance to evaluate the portability of our framework. For
example, we did learn that the datasets generated for traditional computers are
often too large for mobile devices, thus we needed to generate different
datasets for our mobile benchmarks. Also, due the wide range of hardware
resources, a dataset that is suitable for a tablet may be too big for a phone.
Therefore, we will need to have a clever way to select suitable datasets and
make fair comparison between devices.

In term of porting the Parboil benchmarks, we have completed the overall
framework and three benchmarks, namely VectorAdd, Stencil, and Segmm, and close
to finish three more benchmarks, namely, CUTCP, TPACF, MRIQ. The status of the
benchmarks porting are summarized in Table~\ref{table:parboil}. We spent a
significant amount of effort to make the benchmark framework robust and
convenient to work on. For example, the framework contains a Timer utility that
allows us to record time of different segments of each benchmark. We also
implemented a database back-end to store the benchmark results for later
analysis.


\begin{table}[h]\small
\centering
\begin{tabular}{ | l | p{2cm} |}
    \hline 
    Benchmark & Porting status \\ \hline
    Matrix Multiply & \\
    \hspace{0.5cm}-- Java & Completed \\
    \hspace{0.5cm}-- RenderScript & Completed \\ \hline
    Stencil & \\
    \hspace{0.5cm}-- Java & Completed \\
    \hspace{0.5cm}-- RenderScript & Completed \\ \hline
    VectorAdd & \\
    \hspace{0.5cm}-- Java & Completed \\
    \hspace{0.5cm}-- Threaded Java & Completed \\
    \hspace{0.5cm}-- RenderScript & Completed \\ \hline
    CUTCP & \\
    \hspace{0.5cm}-- Java & Completed \\
    \hspace{0.5cm}-- Threaded Java & Partial \\
    \hspace{0.5cm}-- RenderScript & Partial \\ \hline
    MRI-Q & \\
    \hspace{0.5cm}-- Java & Completed \\
    \hspace{0.5cm}-- Threaded Java & Partial \\
    \hspace{0.5cm}-- RenderScript & Partial \\ \hline
    TPACF & \\
    \hspace{0.5cm}-- Java & Completed \\
    \hspace{0.5cm}-- RenderScript & Partial \\ \hline  
    Histogram & Next phase \\ \hline
    Breadth-First Search & Next phase\\ \hline
    MRI-Gridding & Next phase \\ \hline
    Sum of Absolute Differences & Next phase \\ \hline
    Spare-Matrix Vector Multiply & Next phase \\ \hline
    Lattice-Boltzmann Method & TBD \\ \hline
    \hline
\end{tabular}
\caption{Parboil Benchmark Porting Status}
\label{table:parboil}
\end{table}

In term of measurement and analysis, for each implemented benchmark, we compare
three versions, Java, threaded Java, and RenderScript implementations, across
two physical devices: a Samsung Galaxy Nexus phone and a Google Nexus 7 tablet.
The results are detailed in the {\em Analysis} section.

