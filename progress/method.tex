
\section*{Methodology}

In our proposal we stated that if we were able to run OpenCL on the Android devices, then
  we will benchmark against OpenCL.
Sadly, we were not able to find a way to run OpenCL on the devices.
Instead, we opted to implement a serial version of the code in Java along with a multithreaded version
  and compare both against RenderScript.
This is added work, since we now implement 3 versions of the same benchmark, but since we usually
  start with the serial Java version, make it multithreaded, and then port the multithreaded code
  into RenderScript.

All the implementations share a common interface, which means our framework takes the code,
  runs it, and the information is stored in a database on the device which we use for analysis.

\subsection*{RenderScript Kernels}
Currently, the benchmarks we have ported have naive RenderScript kernel
implementations. Our aim has been to write the RenderScript kernels allowing maximum
parallelization and let the RenderScript runtime  perform locality
optimizations (such as, deciding how to group parallel threads when running on a
GPU), if any. We believe that this is the right approach as
the purpose of RenderScript was to keep the programmer oblivious to the low
level hardware details.

We would be working on finding techniques of improving performance of
RenderScript kernels. However, we have realized that it has several challenges.

Firstly, the RenderScript programming environment prevents the standard locality
optimizations by the programmer, such as, tiling. This is primarily because
these optimizations require knowledge of memory sizes, e.g., cache sizes,
scratchpad sizes on GPUs, details which we cannot use as we are not aware of the
processing unit RenderScript kernel is running on.

Secondly, since RenderScript is relatively new, we could not find many examples
of improving performance of RenderScript kernels. Also, not having detailed
documentation on RenderScript makes it difficult to write high performance
RenderScript kernels. For example, there are multiple ways to pass read-only
data to RenderScript runtime. However, we are not sure of the overheads involved
with each approach. With more experience and
experiments, we believe we would have better understanding of such optimizations
and could improve RenderScript performance.

Lastly, RenderScript programming environment has certain idiosyncratic
restrictions which make it difficult to use. For instance, the input and output
{\em Allocation} to a RenderScript kernel have to be of same dimensions. Such
restrictions are counter-intuitive and add to the complexity of using
RenderScript.

\subsection*{Performance Analysis}

We tried to mirror some of the API decisions found in Parboil, this is the same for the timer code.
We implemented a class that allows us to time sections of code and store dump them into the database.

\begin{verbatim}
  timer.start("IO", "Reading inputs");
  float[] inputData = ReadVectorData("input.data");
  timer.stop();
\end{verbatim}

Unlike Parboil's timers however, our interface does not aggregate the data.
This is left for the analysis pass.
The reason is that we feel like Parboil loses some information about which sections are hot in the
  code by aggregating all the data for you.

\subsection*{Power Analysis}

Many hours were wasted trying to find a way to programatically read power data.
A way that may work in debug mode, but other than that it seems like every vendor offers their own
  counter and there is no unified Android API to measure how much power is being drawn at a 
  specific point in time.
If we have extra time, we will revisit this later in the project.

