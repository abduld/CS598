%-----------------------------------------------------------------------------
%
%               Template for sigplanconf LaTeX Class
%
% Name:         sigplanconf-template.tex
%
% Purpose:      A template for sigplanconf.cls, which is a LaTeX 2e class
%               file for SIGPLAN conference proceedings.
%
% Guide:        Refer to "Author's Guide to the ACM SIGPLAN Class,"
%               sigplanconf-guide.pdf
%
% Author:       Paul C. Anagnostopoulos
%               Windfall Software
%               978 371-2316
%               paul@windfall.com
%
% Created:      15 February 2005
%
%-----------------------------------------------------------------------------


\documentclass[nocopyrightspace]{sigplanconf}

% The following \documentclass options may be useful:
%
% 10pt          To set in 10-point type instead of 9-point.
% 11pt          To set in 11-point type instead of 9-point.
% authoryear    To obtain author/year citation style instead of numeric.

\usepackage{natbib}
\usepackage{url}
\usepackage[squaren,thinqspace,binary]{SIunits}

\let\fourth\relax               % Undefine a macro
\let\second\relax               % Undefine a macro
\let\degree\relax               % Undefine a macro
\let\cdot\relax                 % Undefine a macro
\usepackage{array}
\usepackage{amsmath}
\usepackage{amssymb}
\usepackage{graphicx}
%\usepackage{mathabx}
\usepackage{mathtools}
\usepackage{multirow}
\usepackage{bussproofs}
\usepackage{verbatim}
\usepackage{fancyvrb}

\hyphenation{mono-morph-izing}
\hyphenation{mono-morph-ize}
\hyphenation{mono-morph-ic}
\hyphenation{poly-morph-ic}
\hyphenation{geo-mean}

%------------------------------------------------------------------------------
% Global macro definitions

\newcommand{\abs}[1]{\lvert#1\rvert}
\newcommand{\proportional}{\mathbin{\propto}}
\newcommand{\inst}{\mathrel{\succeq}}
\newcommand{\defeq}{\mathrel{:=}}
\newcommand{\defeqcont}{\mathrel{\hphantom{\defeq}\mathllap{\mid}}}
\newcommand{\overlineX}[1]{\overline{\vphantom{\beta}#1}}
\newcommand{\overlinet}[1]{\overline{\vphantom{t}#1}}
\newcommand{\conexprX}[3]{#1\app\overlineX{#2}\app\overlineX{#3}}
\newcommand{\conexpr}[3]{#1\app\overline{#2}\app\overline{#3}}
\newcommand{\dataexpr}[4]{#1\app\overline{#2}\app\overline{#3}\app\overline{#4}}
\newcommand{\subst}[2]{[#2/#1]}
\newcommand{\substo}[2]{[\overline{#2/#1}]}

\newcommand{\ztype}[1]{\ensuremath{#1_{\text{type}}}}
\newcommand{\name}[1]{\ensuremath{\mathsf{#1}}} % Object functions
\newcommand{\meta}[1]{\ensuremath{\mathrm{#1}}} % Meta-functions
\newcommand{\bmk}[1]{\texttt{#1}}
\newcommand{\app}{\;}

\def\T{\mathbin{::}}

\newcommand{\qforall}[1]{\forall#1.\nolinebreak[1]\:}
\newcommand{\qexists}[1]{\exists#1.\nolinebreak[1]\:}
\newcommand{\qlambda}[1]{\lambda#1 \to}
\newcommand{\qLambda}[1]{\Lambda#1 \to}
\newcommand{\qforallT}[2]{\qforall{#1\T#2}}
\newcommand{\qexistsT}[2]{\qexists{#1\T#2}}
\newcommand{\qlambdaT}[2]{\qlambda{#1\T#2}}
\newcommand{\qLambdaT}[2]{\qLambda{#1\T#2}}

\newcommand{\letE}[1]{\text{\bf let}\;#1\;\text{\bf in}\;}
\newcommand{\letrecE}[1]{\text{\bf letrec}\;#1\;\text{\bf in}\;}
\newcommand{\letfE}[1]{\text{\bf letfun}\;#1\;\text{\bf in}\;}
\newcommand{\caseE}[1]{\text{\bf case}\;#1\nolinebreak[1]\;\text{\bf of}\;}
\newcommand{\dataE}[1]{\text{\bf data}\nolinebreak[1]\;#1\nolinebreak[0]\;\text{\bf where}\;}
\newcommand{\classE}[1]{\text{\bf class}\nolinebreak[1]\;#1\nolinebreak[0]\;\text{\bf where}\;}
\newcommand{\ifE}[3]{\ifO#1\;\thenO#2\;\elseO#3}
\newcommand{\externO}{\text{\bf extern}\;}
\newcommand{\caseO}{\text{\bf case}\;}
\newcommand{\dataO}{\text{\bf data}\;}
\newcommand{\classO}{\text{\bf class}\;}
\newcommand{\ifO}{\text{\bf if}\;}
\newcommand{\thenO}{\text{\bf then}\;}
\newcommand{\elseO}{\text{\bf else}\;}
\newcommand{\whereO}{\text{\bf where}\;}
\newcommand{\ofO}{\text{\bf of}\;}
\newcommand{\letO}{\text{\bf let}\;}
\newcommand{\letrecO}{\text{\bf letrec}\;}
\newcommand{\inO}{\text{\bf in}\;}
\newcommand{\typeO}{\text{\bf type}\;}

\def\kindstar{\ensuremath{\mathord{\star}}}
\def\kindZ{\ensuremath{\mathbb{Z}}}
\def\kindbox{\name{box}}
\def\kindbare{\name{bare}}
\def\kindval{\name{val}}
\def\kindout{\name{mut}}
\def\kindinit{\name{init}}


\def\typeOut{\name{Mut}}
\def\typeStore{\name{Store}}
\def\typeInit{\name{Init}}
\def\typeStored{\name{Stored}}
\def\typeBoxed{\name{Boxed}}
\def\typeAsBox{\name{AsBox}}
\def\typeAsBare{\name{AsBare}}
\def\typeArr{\name{Arr}}
\def\typeRef{\name{Ref}}
\def\typeRep{\name{Rep}}
\def\typeTuple{\name{Tuple}}
\def\typeTupleV{\name{TupleV}}
\def\typeList{\name{List}}
\def\typeInt{\name{Int}}
\def\typeFloat{\name{Float}}
\def\typeBool{\mathbf{2}}
\def\typeMaybe{\name{Maybe}}
\def\typeScatter{\name{Sc}}
\def\typeSz{\name{Sz}}
\def\typeSv{\name{Sv}}
\def\typeZ{\name{Z}}
\def\typePtr{\name{Ptr}}

\def\contuple{\name{tuple}}
\def\constored{\name{stored}}
\def\conboxed{\name{boxed}}
\def\conref{\name{ref}}
\def\conrepr{\name{rep}}
\def\consize{\name{size}}
\def\conasBare{\name{asBare}}
\def\conasBox{\name{asBox}}
\def\concopy{\name{copy}}
\def\conreprInt{\name{repInt}}
\def\conreprTuple{\name{repTuple}}
\def\coniint{\name{z}}
\def\coneiint{\name{exz}}
\def\consz{\name{sz}}
\def\consv{\name{sv}}
\def\conz{\name{z}}
\def\contrue{\name{true}}
\def\confalse{\name{false}}

\newcommand{\Unit}{\langle\mspace{0.75mu}\rangle}
\newcommand{\Prod}[1]{\langle#1\rangle}
\newcommand{\Array}[1]{\mathopen{\text{\sf[}}#1\mathclose{\text{\sf]}}}

\def\iterIdx{\name{Idx}}
\def\iterStep{\name{Step}}
\def\iterFold{\name{Fold}}
\def\iterColl{\name{Coll}}
\def\iterIter{\name{Iter}}

\newcommand{\todo}[1]{\textbf{[#1]}}

\def\proofsep{13pt}

\newcommand{\proofspacing}{%
  \parindent=0cm%
  \everypar={\hskip 0cm plus 1fill}%
  \parfillskip=0cm plus 1fill\relax%
  \parskip=\proofsep}

\newenvironment{inlinecodeexample}%
{\begin{tabbing}}
{\end{tabbing}}

\newcommand{\setwherestretch}{\def\arraystretch{1.15}}
\newcommand{\where}{\hphantom{=\mathord{}}\mathllap{\text{where}}}
\newenvironment{whereblock}[1]%
{\setwherestretch\where\begin{array}[t]{#1}}%
{\end{array}}

%------------------------------------------------------------------------------

\begin{document}

%\titlebanner{banner above paper title}        % These are ignored unless
%\preprintfooter{preprint}   % 'preprint' option specified.

\title{Performance Analysis of RenderScript Progress}
%\subtitle{}

\authorinfo{Abdul Dakkak \and Cuong Manh Pham \and Prakalp Srivastava}
           {University of Illinois at Urbana-Champaign}
           {\{dakkak, pham9, psrivas2\}@illinois.edu}

\conferenceinfo{CONF 'yy}{Month d--d, 20yy, City, ST, Country} 
\copyrightyear{20yy} 
\copyrightdata{978-1-nnnn-nnnn-n/yy/mm} 
\doi{nnnnnnn.nnnnnnn}

\maketitle

%\category{CR-number}{subcategory}{third-level}

%\terms
%term1, term2

%\keywords
%keyword1, keyword2

\section*{Progress Summary}

This section summarizes our progress to date~\footnote{Code is hosted on
\url{https://github.com/cmpham/RSBench}}. Table~\ref{table:parboil} shows the
porting status of each version of the benchmarks in the Parboil Benchmark Suite.
In this phase, we also worked on improving the framework, making it easy to collect
data and plot graphs. In parallel, we worked on getting OpenCL to work on our
devices (this was required as Android does not have official support for OpenCL
on Nexus tablets), and started porting the OpenCL version of the benchmarks. 
Further information on each task
is described in the subsequent sections.


\begin{table}[h]\small
\centering
\begin{tabular}{ | l | c | c | c | c |}
    \hline 
    Benchmark & \multicolumn{4}{|c|}{Porting status} \\ \cline{2-5}
              & Java & T-Java & OpenCL & RS \\ \hline
    Matrix Multiply & C & C & C & C \\ \hline
    Stencil & C & C & I & C \\ \hline
    VectorAdd & C & C & I & C \\ \hline
    CUTCP & C & P  & I  & P \\ \hline
    MRI-Q & C & C & P & C \\ \hline
    TPACF & C & C & I & I \\ \hline
    Histogram & C & C & I & I \\ \hline
    Breadth-First Search & C & I & I & I \\ \hline
    MRI-Gridding & I & I & I & I \\ \hline
    Sum of Absolute Differences & P & I & I & I \\ \hline
    Spare-Matrix Vector Multiply & I & I & I & I \\ \hline
    Lattice-Boltzmann Method & \multicolumn{4}{|c|}{TBD} \\ \hline
    \hline
\end{tabular}
\caption{Parboil Benchmark Porting Status. T-Java: Threaded Java; RS:
RenderScript; C: Completed; I: Incomplete; P: Partial}
\label{table:parboil}
\end{table}


\section*{Methodology}

There is one major change in our strategy to evaluate RenderScript from our last
progress report. In our last report we mentioned that because of the ongoing competition between
Google's RenderScript and Khronos Group's OpenCL, Google has decided not to ship
GPU-CPU RenderScript drivers in their Nexus phones/tablets firmware. They never
officially supported the drivers for use by developers, but the Android 4.2 was
found to have unofficial OpenCL drivers. It was predicted that these drivers
would be removed from future versions and as expected Google removed these drivers from
Android 4.3 onwards. As a result we decided to focus on evaluating RenderScript
implementation against Single threaded and multi-threaded Java implementations. 

However, we have been able to manually install OpenCL drivers in Android 4.4.2
running on our Nexus 7 device and we would now be able to evaluate
RenderScript against OpenCL.
This requires one to hack the device and can be done for the Nexus 5 device, but
  we have not done so already.


\subsection*{Performance Analysis}

We use the timer discussed in the previous section as our means to measure performance.
For each implementation, we run the compute section multiple times taking the 
  minimum time across runs.
We carefully place hints to Java so that it performs the garbage collection before
  it enters the compute part of code --- this results in the Java garbage collector not being invoked during the compute part of our code by chance.

While we did not do so in our current measures, we plan on making sure
   we exit all applications before running our benchmark suite.
While we do not think that this to be a major factor, we will err on the side of caution.

\subsection*{Power Analysis}

No progress has been made in trying to figure out the APIs needed to be called
  to perform the power analysis.
We may resort to some tools to run the power analysis, but given that the majority
  of the runtime is devoted to I/O, we need to understand how to do power
  analysis without accounting for the I/O time.


\section*{Analysis}
\label{sec:analysis}

The emulator is used to perform the debugging (although it does not seem to be able to run RenderScript code).
To run the code, we are using two devices for the analysis.
The first is a Nexus 7 with the following specs:

\begin{verbatim}
    CPU: Qualcomm Snapdragon S4 Pro, 1.5GHz
    GPU: Adreno 320, 400MHz
    Memory - 2 GB
    Storage 32 GB
\end{verbatim}

The second is a Samsung Galaxy Nexus with the following specs:

\begin{verbatim}
    CPU: ARMv7, 2 cores, 1200 Mhz, SIMD NEON
    Memory: 694, JVM max: 96 MB
    GPU: PowerVR SGX 540.
\end{verbatim}


\subsection*{Perliminary Results}

While we do have implementations for 3 out of the 4 benchmarks we promised in the proposal schedule,
  our results only show part of those.
The data has not been fully analyzed, but we can give some perliminary insights into the expected
  results.


\begin{figure}[t!]
\includegraphics[scale=0.125]{VectorAdd.png}
\caption{VectorAdd Benchmark.}
\label{fig:vecadd}
\centering
\end{figure}



\begin{figure}[t!]
\includegraphics[scale=0.125]{Sgemm.png}
\caption{Matrix Matrix Multiplication Benchmark.}
\label{fig:sgemm}
\centering
\end{figure}



In figure~\ref{fig:stencil}, we should the results from running the stencil code across the different
  implementations.

The serial versions have an added {\tt RSSetup} time, this is either due to bad parsing of the 
  output data or an incorrect labeling of a timer in the source code.

\begin{figure}[t!]
\includegraphics[scale=0.125]{Stencil.png}
\caption{Stencil Benchmark.}
\label{fig:stencil}
\centering
\end{figure}


While these graphs have the infromation, they are not presented in the clearest way.
Future work would remove the IO times from the plots, this would avoid us having to use Log scalling
  and would make some of the differences more visable.
  


\section*{Complications}

Our main complication came from the fact that we have never developed Android applications
  before.
We approached it as if we were developing Java applications, but quickly realized that 
  there are some constraints placed by the Android environment.

Furthermore, because the Android software and hardware is fragmented, references we found
  on how to solve certain problems either did not pertain to the hardware we were using
  or the Android version we are targeting.
This is the main reason why we were not able to capture power information from the device,
  since each device has a different file system path that stores those counters.

Finally, the datasets used by Parboil each have a different format.
This means that for each benchmark we have to write a file reader and writer to read in the proper benchmarks.
The datasets are also quite large --- a few GB.
At first, we were bundling the data with the application which resulted in the Android application
  to balloon into a 1GB application.
We have now moved the data to be read from the SD card.
One possible alternative is to generate random data (according to some distribution).
While this is simple for benchmarks such as matrix multiplication, it is not clear how one would
  generate meaningful data for the MRI benchmark.

\section*{Schedule}

As before, we think we are not too far behind of our initial 
  projection while still adding new tasks to our initial proposal.
We had setup around 1-2 weeks to allow us to recover 
  from some set backs.
In the comming weeks we will be concentrating on finishing the 
  RenderScript implementations for all the benchmarks as well
  as resolving the OpenCL issues.
To allow us to stay on schedule, we may be dropping some benchamrks that
  are either too complicated, or cannot be represented easily in Java 
  (some benchmarks make extensive use of Macros, for example).
For the time being, we still do not forsee time to implement the LBM benchmark.


% We recommend abbrvnat bibliography style.

% \bibliographystyle{abbrvnat}
% \bibliography{paper}

% The bibliography should be embedded for final submission.

%\begin{thebibliography}{}
%\softraggedright
%
%\bibitem[Smith et~al.(2009)Smith, Jones]{smith02}
%P. Q. Smith, and X. Y. Jones. ...reference text...
%
%\end{thebibliography}

\end{document}
