
\section{Methodology}



\subsection{Performance Analysis}

Parboil is unique in the it offers optimized benchmark implementations for different targets.
Benchmarks have been ported for multicore configurations using {\tt C/OpenMP}, 
	GPU implementations using {\tt CUDA} and {\tt OpenCL} exist as well.
There is currently no way to run {\tt CUDA} on Android devices, but OpenCL 2.0~\cite{OpenCL} does
	have support for \textit{Android Installable Client Driver Extension}.
If OpenCL is supported by Android, then we will do performance measurements using it as a base line.
If not, then we will use a native client multithreaded C implementation as our base line.

\subsection{Power Analysis}

Aside from performance, power is an important factor for mobile applications.
Based on a quick survey, there does seem to be power profiling tools for android~\cite{Google:2014:Power}.
We will be looking at how the change in language impacts the power usage of the device.

\subsection{Programability}

The final (objective) metric that we will look at is programability.
Performance programming using CUDA or OpenCL exposes low level details,
	the memory heirarchy being the most noticable, for hardware.
This makes programs written in these languages sensitive to these low
	level optimizations as they are run on different hardware.
Renderscript hides low level details and while promising to be performance
	portable.
In the process of porting the Parboil benchmarks, we will
	examine these claims.

\subsection{Hardware}

While development will be against the Android emulator, our performance
	metrics will be measrured on real hardware.
Currently we have access to a Nexus 7 Android tablet, but we haved contacted
	the Android development team to find if there is a way to get
	access to more devices to run our benchmarks.

Another approach to consider is using existing services,
	such as Apkudo~\cite{apkudo}, that allow one to test Android
	applications across devices.
The Apkudo service allows one to upload an application and then run it on 
	over 260 android devices.

