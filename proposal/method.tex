
\section{Methodology}

This project study will analyze RenderScript based on three criteria: performance, power, and programmability.
The following sections describe each one of the analysis in detail.


\subsection{Performance Analysis}

In general, a Parboil benchmark has 3 parts: it reads the data, runs the kernel, and checks if the 
    computed solution is valid.
We will be concentrating on the latter 2 parts of each benchmark and comparing them against other
    targets supported by Parboil.
Since there is currently no way to run {\tt CUDA} on Android devices, we will be concentrating
    on comparing RenderScript's performance against the OpenCL implementation.
While OpenCL 2.0~\cite{OpenCL} does
	have support for \textit{Android Installable Client Driver Extension}, it is still not clear whether 
    it is an advertised feature and/or it would run on the Android device we have.
If OpenCL is supported, then we will do performance measurements using it as a base line.
If not, then we will use a native multithreaded C implementation as our base line.

\subsection{Power Analysis}

Aside from performance, power is an important factor for mobile applications.
Based on our research, there does seem to be power profiling tools for Android~\cite{Google:2014:Power}.
Using the power profiling tools, we will be looking at how the change in
    implementation impacts the power usage of the device.

\subsection{Programability}

The final (objective) metric that we will look at is programmability.
As discussed previously, performance programming using CUDA or OpenCL
    exposes low level details of the hardware --- the memory hierarchy being the most noticeable.
The performance of programs written in these languages is very sensitive to low
	level optimizations.
This results in codes that performs well on one architecture (or one generation of an architecture)
    but poorly on another architecture.
RenderScript makes code more programmable (by hiding low level details) while promising to be performance
	portable.
In the process of porting the Parboil benchmarks, we will
	examine the ease of programmability claims.

\subsection{Hardware}

While development will be against the Android emulator, our performance
	metrics will be measured on real hardware.
Currently we have access to a Nexus 7 Android tablet, but we have contacted
	the Android development team to find if there is a way to get
	access to more devices to run our benchmarks.

One approach we will consider is using existing services,
	such as Apkudo~\cite{apkudo}, that allow one to test Android
	applications across devices.
The Apkudo service allows one to upload an application and then run it on 
	over 260 android devices.
One concern we have is whether we can get accurate timing information from
    such public services.

