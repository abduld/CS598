\section{Introduction}

Heterogeneous computing promises to address the rising power dissipation problem
of today's traditional, homogeneous multi-core
systems. It provides the ability to integrate a variety of processing elements,
such as large and small general purpose cores, GPUs, DSPs, and custom or
semi-custom hardware into a single system. If applications can efficiently use
the full range of available hardware, it can provide significant energy
savings over conventional processors by executing portions of the code on the
device which best suits it. This has led mobile devices such as smartphones
and tablets, which deal with a variety of applications with limited battery
life, to move towards heterogeneous designs.

However, heterogenity of hardware resouces also has led to a diverse landscape
of different programming models, run-time systems, profiling and debugging tools
for application development. The differences are so deep that programmers are
often experts on only one class of device, e.g., an expert GPU programmer will
not have much DSP expertise and vice-versa. This is highly inefficient and
unproductive: we cannot expect applications to use a separate language for each
class of compute unit. If we want applications to use the full range of
available hardware to maximize perofrmance or energy efficiency or both, the
programming environment has to provide common abstractions for available
hardware compute units.

%The industry and the reseach community in this field have been trying to solve
%this problem. 

%The current programming approaches that are portable such as
%OpenCL but at a low-level of abstraction as with,
%e.g., OpenCL. OpenCL requires explicit coding of data transfer, memory
%management, kernel launch etc., and re-optimization and adaptation to obtain
%decent performance when migrating to a new configuration. 
%Programmability of Heterogeenous Systems

The industry and the research community have been trying to solve this problem.
The recent development of Renderscript~\cite{Renderscript}  

Renderscript

What Renderscript promises

What do we want to study?


