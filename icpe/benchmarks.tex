\section{Benchmarks}
\label{sec:benchmarks}

We extend the Parboil benchmark suite to run on Android devices.
Table~\ref{table:parboil} shows the porting status of each version of the
benchmarks in the Parboil Benchmark Suite.


\begin{table}
\centering
\begin{tabu}{ | l | c | c | c | c | c | c | c | c |}
    \hline 
    Name & \multicolumn{8}{c|}{Implementations} \\ \cline{2-9}
                      		   & NC & MP  & J    & JT     & CL     & RS & MR & TB\\ \hline
    \textbf{VecAdd}            & C  & C   & C    & C      & C      & C  & C    & C\\ \hline
    \textbf{MM}                & C  & C   & C    & C      & C      & C  & C    & C\\ \hline
    \textbf{Stencil}           & C  & C   & C    & C      & C      & C  & C    & C\\ \hline
    \textbf{CUTCP}             & N  & N   & C    & C      & C      & C  & N    & N\\ \hline
    \textbf{MRIQ}              & N  & N   & C    & C      & C      & C  & N    & N\\ \hline
    \textbf{TPACF}             & B  & B   & C    & C      & C      & C  & N    & N\\ \hline
    \textbf{Histo}             & C  & B   & C    & C      & C      & C  & N    & N\\ \hline
    \textbf{BFS}               & \multicolumn{8}{c|}{N} \\ \hline
    \textbf{MRIG}              & \multicolumn{8}{c|}{N} \\ \hline
    \textbf{SAM}               & \multicolumn{8}{c|}{N} \\ \hline
    \textbf{SPMV}              & \multicolumn{8}{c|}{N} \\ \hline
    \textbf{LBM}               & \multicolumn{8}{c|}{N} \\ \hline
    \hline
\end{tabu}
\caption{Parboil Benchmark Porting Status. \textbf{NC}: Native C; \textbf{MP}:
Native C with OpenMP; \textbf{JT}: Threaded Java; \textbf{CL}: OpenCL;
\textbf{RS}: RenderScript; \textbf{MR}: MARE; \textbf{TB}: Thread Building Blocks; \textbf{C}: Completed; \textbf{N}: No Implementation;
\textbf{B}: a bug causes the benchmark to crash.}
\label{table:parboil}
\end{table}

In this section, we give an overview of the benchmarks implemented along with 
	the dataset sizes we used when profiling the results.
While the Parboil benchmark suite represents scientific workloads, we expect them to be 
	representative of future Android applications --- given that we already 
	see laptops using Android as their OS.

\subsection{VectorAdd}

The VectorAdd benchmark adds two floating point vectors with $8K$ elements.
Compared to other benchmarks, VectorAdd has a very high memory to compute ratio.
The benchmark is therefore not a fit for parallelization, but we use it to examine
  the overhead behavior.

\subsection{SGEMM}

The SGEMM benchmark multiplies two matrices $A$ and $B$ produces an output $C$.
Matrix $A$ is of dimension $128 \times 96$ and $B$ is of dimension $96 \times 160$.
Matrix $A$ is stored in the row major format, while $B$ is stored in the column major format ---
	we therefore do not need to transpose $B$ to make effective use of the cache.

The OpenMP code uses the \fix{\#pragma omp parallel for shared(A, B, C) collapse(2)}
	pragma, in which, based on the processor utilization,
	the Android compiler was not able to parse as valid OpenMP code.
Therefore, the OpenMP code for SGEMM is equivalent to the serial C code.

\subsection{TPACF}

The TPACF benchmark analyzes the angular distribution of astronomical objects.
The algorithm computes the distances between all pairs of coordinates in a dataset
	and then performs histogramming.
The results are collected in three histograms which are then cross-correlated to find
	the statistical spatial distribution of the astronomical bodies.
For our analysis, we use $100$ datasets, each containing $487$ coordinates.

\subsection{MRIQ}

The MRIQ benchmark computes a non-uniform 3D inverse-Fourier-transform, representing a calibration matrix.
The calibration matrix is used to perform 3D image reconstruction from MRI data which is 
	presented in a non-Cartesian space.
The input dataset is of size $32 \times 32 \times 32$ containing trajectory information in $3$D
	as direction parameter in $2$D.

\subsection{Stencil}

The Stencil benchmark computes a 7 point stencil of an input volume. 
The input volume has dimension $128 \times 128 \times 32$.
Each kernel performs a standard 7 point stencil: accessing the $6$ adjacent voxels,
	scaling and then adding them to the current voxel.
The result is then placed in the output buffer.

\subsection{Histogram}

The Histogram benchmark computes the histogram of an input image.
The input image is of size $996 \times 1040$ and compute a 
	histogram of size $256$.
Each bin in the histogram saturates at the value $255$.

Unlike the other parallel implementations, which use atomics, both the
	threaded Java and OpenMP implementations use privatizations.
Private histogram copies are allocated, each thread 
	operates on its own private copy, and once the threads finish, the
	master thread aggregates the results.

