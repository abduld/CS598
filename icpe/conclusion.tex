\section{Conclusion}
\label{sec:conclusion}

Unlike CUDA or OpenCL which are specifically marketed to get maximal performance, 
RenderScript markets itself as a language that would give good performance across devices.
For performance portability, while maintaining the semantics and reducing energy utilization,
	we believe that RenderScript delivers on its promise.

Furthermore, unlike the C or OpenMP which rely on static analysis and compiler
	intelligence to get free speedups, and OpenCL which is not officially supported,
	we see future versions of RenderScript offering free speedups for programmers.
And while we would like RenderScript to expose more information about the running hardware,
	this is because we are more familiar with CUDA and OpenCL programming where such facilities
	are available.
We think that we are in the minority, and most programmers do not need full control over the hardware.
Furthermore, exposing some hardware detail would complicate the language and would cause
	more overhead by the runtime.
This entices us to suggest RenderScript as the language, over C or OpenMP, when writing compute intensive
	codes on Android devices.
