\section*{Complications}
\subsection*{Running OpenCL Benchmarks}

%This was a major challenge for us, but we have been
%able to successfully insall these drivers manually on our Nexus 7 tablets.

While we have been successful in getting OpenCL to run on Nexus 7 tablets
running Android 4.4.2 (Kit-Kat), the OpenCL drivers we installed for Adreno
GPU were from
Android 4.2 and that too were not part of the official release. Thus, they might
be unstable. The parboil benchmark suite is written using the OpenCL C API while
we have been able to run the SGEMM benchmark and an example application only by using
the OpenCL C++ Wrapper API. All our attempts to run the parboil benchmark suite as
is, using the OpenCL C API have failed so far. Any call to native JNI code
ends up corrupting the C++ heap and thus, crashing the application. So far, we
have only been successful in running SGEMM by porting the parboil OpenCL C
API version to OpenCL C++ Wrapper API version manually.

Another challenge is to tune these parboil OpenCL benchmarks for the Adreno GPU.
The current versions of these benchmarks are tuned for discrete Nvidia GPUs on desktop and might
suffer poor performance while running on the Adreno GPU of Nexus 7, which is much
smaller.

Both these challenges significantly
increase our work as any OpenCL parboil benchmark needs to be ported to OpenCL
C++ Wrapper API version to run on the Nexus 7 tablets, and has to be tuned for
the Adreno GPU. We did not foresee the former
challenge initially and thus we are now contemplating on implementing only a
subset of benchmarks for OpenCL.

\subsection*{Visualization}

We have yet to figure out an easy to read way to represent our results.
Our current visualizations are easier to read than before, but they
  still encode a lot of information that is hard to read.
We plan on reviewing how people present their graphs for similar sorts
  of problems for the final report.
